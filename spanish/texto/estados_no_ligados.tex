% texto/estados_no_ligados.tex

\chapter{Estados no ligados}
Los estados no ligados o enlazados son aquellos en los que una partícula
es libre, su energía total es mayor o igual a la energía potencial que rodea
al pozo; en este caso $E\ge 0$; en términos de variables reducidas, $\epsilon_r \ge 1$.
En un estado enlazado la partícula es libre para moverse por todo el espacio.

\subsection{Primeros pasos en la resolución}
\subsubsection{Zona I}
\noindent
Dominio: $x_r < 0$.
Ecuación de Schrödinger reducida:
$\psi''_{rI} - 2(1-\epsilon_r)\,\psi_{rI} = 0$.
Ecuación característica usando $\psi_{rI}=e^{sx_r}$:
\[
s^2-2(1-\epsilon_r)=0
\]
\[
s^2=2(1-\epsilon_r)
\]
\[
s = \pm\sqrt{2(1-\epsilon_r)}
\]

Así:
\[
\psi_{rI}(x_r)
 = C_1\,e^{-\sqrt{2\epsilon_r}\,x_r} + C_2\,e^{\sqrt{2\epsilon_r}\,x_r}
\]

La partícula puede moverse libremente por todo el espacio, por lo que la
función de onda no tiene que anularse cuando se aleja.

\subsubsection{Zona II}
\noindent
Dominio: $0 < x_r < l_r$.\\
Ecuación de Schrödinger reducida: $\psi''_{rII} - 2\epsilon_r\psi_{rII}= 0$.\\
Ecuación característica usando $\psi_{rII}=e^{sx_r}$:
\[
s^2+2\epsilon_r=0
\]
\[
s^2=-2\epsilon_r
\]
\[
s = \pm i\sqrt{2\epsilon_r}
\]

Así:
\begin{align*}
\psi_{rII}(x_r)
&=C_3\,e^{-i\,\sqrt{2\epsilon_r}\,x_r} + C_4\,e^{+i\,\sqrt{2\epsilon_r}\,x_r}\\
&=C_3\,\cos\left(\sqrt{2\epsilon_r}\,x_r\right)
 - i C_3\,\sen\left(\sqrt{2\epsilon_r}\,x_r\right)
 + C_4\,\cos\left(\sqrt{2\epsilon_r}\right)\,x_r 
 + i C_4\,\sen\left(\sqrt{2\epsilon_r}\,x_r\right)\\
&=(C_3+C_4)\,\cos\left(\sqrt{2\epsilon_r}\,x_r\right)
 + i (C_4-C_3)\,\sen\left(\sqrt{2\epsilon_r}\,x_r\right)
\end{align*}

La función de onda en la zona II es:
\[
\psi_{rII}(x_r)
= A\,\cos\left(\sqrt{2\epsilon_r}\,x_r\right)
 + B\,\sen\left(\sqrt{2\epsilon_r}\,x_r\right)
\]

\subsubsection{Zona III}
\noindent
Dominio: $x_r > l_r$.\\
Ecuación de Schrödinger reducida:
$\psi''_{rIII} - 2(1-\epsilon_r)\psi_{rIII}= 0$.\\
Ecuación característica usando $\psi_{rIII}=e^{sx_r}$:
\[
s^2-2(1-\epsilon_r)=0
\]
\[
s^2=2(1-\epsilon_r)
\]
\[
s = \pm\sqrt{2(1-\epsilon_r)}
\]

Así:
\[
\psi_{rIII}(x_r)
 = C_5\,e^{-\sqrt{2(1-\epsilon_r)}\,x_r} + C_6\,e^{\sqrt{2(1-\epsilon_r)}\,x_r}
\]

\subsection{Condiciones límite}

Las funciones de onda en las zonas I y II deben tender al mismo valor en
$x_r=0$.
\[
\lim_{x_r\to\,0} \psi_{rI}(x_r)
=
\lim_{x_r\to\,0} \psi_{rII}(x_r)
\]

Así:
\[
C_2 = A
\]

Las derivadas de las funciones de onda en I y en II también deben tender al
mismo valor en $x_r=0$:
\[
\lim_{x_r\to\,0} \psi'_{rI}(x_r)
=
\lim_{x_r\to\,0} \psi'_{rII}(x_r)
\]

Calculemos las derivadas de las funciones de onda en las zonas I y II:
\begin{align*}
\psi'_{rI} &= \frac{d\psi_{rI}(x_r)}{dx_r}
  = C_2\,\sqrt{2(1-\epsilon_r)}\,e^{\sqrt{2(1-\epsilon_r)}\,x_r}
  = A\,\sqrt{2(1-\epsilon_r)}\,e^{\sqrt{2(1-\epsilon_r)}\,x_r}\\
\psi'_{rII} &= \frac{d\psi_{rII}(x_r)}{dx_r}
  = -A\,\sqrt{2\epsilon_r}\,\sen\left(\sqrt{2\epsilon_r}\,x_r\right)
  + B\,\sqrt{2\epsilon_r}\,\cos\left(\sqrt{2\epsilon_r}\,x_r\right)
\end{align*}

Luego:
\[
A\,\sqrt{2(1-\epsilon_r)} =  B\,\sqrt{2\epsilon_r}
\]
\[
B = \sqrt{\frac{1-\epsilon_r}{\epsilon_r}}\,A
\]

Las funciones de onda en las zonas II y III deben tender al mismo valor en
en el otro extremo del pozo de potencial, $x_r=l_r$:
\[
\lim_{x_r\to\,l_r} \psi_{rII}(x_r)
=
\lim_{x_r\to\,l_r} \psi_{rIII}(x_r)
\]

\[
A\,\cos\left(l_r\sqrt{2\epsilon_r}\right)
 + A\,\sqrt{\frac{1-\epsilon_r}{\epsilon_r}}
   \,\sen\left(l_r\,\sqrt{2\epsilon_r}\right)
=
C_5\,\,e^{-l_r\sqrt{2(1-\epsilon_r}}
\]

Despejando $C_5$:
\begin{equation}
\label{c5_1}
C_5 = A\,e^{l_r\sqrt{2(1-\epsilon_r)}}
  \left\lbrace\cos\left(l_r\sqrt{2\epsilon_r}\right) 
  + \sqrt{\frac{1-\epsilon_r}{\epsilon_r}}\,
         \sen\left(l_r\sqrt{2\epsilon_r}\right)\right\rbrace
\end{equation}


