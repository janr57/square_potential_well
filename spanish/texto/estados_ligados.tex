% texto/estados_ligados.tex

\section{Estados enlazados}
Los estados ligados o enlazados son aquellos en los que una partícula
no es libre, se encuentra en el interior del pozo de potencial y su energía
total es inferior a la energía potencial que rodea al pozo; en este caso
$E<U_0$; en términos de variables reducidas, $\epsilon_r < 1$.
En un estado enlazado la partícula no podría escapar del pozo según la Física
clásica.

\subsection{Primeros pasos en la resolución}
\subsubsection{Zona I}
\noindent
Dominio: $x_r < 0$.
Ecuación de Schrödinger reducida:
$\psi''_{rI} - 2(1-\epsilon_r)\,\psi_{rI} = 0$.
Ecuación característica usando $\psi_{rI}=e^{sx_r}$:
\[
s^2-2(1-\epsilon_r)=0
\]
\[
s^2=2(1-\epsilon_r)
\]
\[
s = \pm\sqrt{2(1-\epsilon_r)}
\]
So:
\[
\psi_{rI}(x_r)
 = C_1\,e^{-\sqrt{2\epsilon_r}\,x_r} + C_2\,e^{\sqrt{2\epsilon_r}\,x_r}
\]

Primera condición límite, la función de onda debe anularse en $-\infty$:
\[
\lim_{\scriptsize x_r\to\, -\infty} \psi_{rI}(x_r) = 0
\]

Deducimos que $C_1$ debe valer cero. La función de onda en la zona I es:
\[
\psi_{rI}(x_r) = C_2\,e^{\sqrt{2(1-\epsilon_r)}\,x_r}
\]

\subsubsection{Zona II}
\noindent
Dominio: $0 < x_r < l_r$.\\
Ecuación de Schrödinger reducida: $\psi''_{rII} - 2\epsilon_r\psi_{rII}= 0$.\\
Ecuación característica usando $\psi_{rII}=e^{sx_r}$:
\[
s^2+2\epsilon_r=0
\]
\[
s^2=-2\epsilon_r
\]
\[
s = \pm i\sqrt{2\epsilon_r}
\]

Así:
\begin{align*}
\psi_{rII}(x_r)
&=C_3\,e^{-i\,\sqrt{2\epsilon_r}\,x_r} + C_4\,e^{+i\,\sqrt{2\epsilon_r}\,x_r}\\
&=C_3\,\cos\left(\sqrt{2\epsilon_r}\,x_r\right)
 - i C_3\,\sen\left(\sqrt{2\epsilon_r}\,x_r\right)
 + C_4\,\cos\left(\sqrt{2\epsilon_r}\right)\,x_r 
 + i C_4\,\sen\left(\sqrt{2\epsilon_r}\,x_r\right)\\
&=(C_3+C_4)\,\cos\left(\sqrt{2\epsilon_r}\,x_r\right)
 + i (C_4-C_3)\,\sen\left(\sqrt{2\epsilon_r}\,x_r\right)
\end{align*}

La función de onda en la zona II es:
\[
\psi_{rII}(x_r)
= A\,\cos\left(\sqrt{2\epsilon_r}\,x_r\right)
 + B\,\sen\left(\sqrt{2\epsilon_r}\,x_r\right)
\]

\subsubsection{Zona III}
\noindent
Dominio: $x_r > l_r$.\\
Ecuación de Schrödinger reducida:
$\psi''_{rIII} - 2(1-\epsilon_r)\psi_{rIII}= 0$.\\
Ecuación característica usando $\psi_{rIII}=e^{sx_r}$:
\[
s^2-2(1-\epsilon_r)=0
\]
\[
s^2=2(1-\epsilon_r)
\]
\[
s = \pm\sqrt{2(1-\epsilon_r)}
\]
Así:
\[
\psi_{rIII}(x_r)
 = C_5\,e^{-\sqrt{2(1-\epsilon_r)}\,x_r} + C_6\,e^{\sqrt{2(1-\epsilon_r)}\,x_r}
\]

Teniendo en cuenta que la función de onda se debe anular en $+\infty$:
\[
\lim_{\scriptsize x_r\to\, +\infty} \psi_{rIII}(x_r) = 0
\]

Deducimos que $C_6$ debe valer cero. La función de onda en la zona III es:
\[
\psi_{rIII}(x_r) = C_5\,e^{\sqrt{2(1-\epsilon_r)}\,x_r}
\]

\subsection{Otras condiciones límite}

Las funciones de onda en las zonas I y II deben tender al mismo valor en
$x_r=0$.
\[
\lim_{\scriptsize x_r\to\,0} \psi_{rI}(x_r)
=
\lim_{\scriptsize x_r\to\,0} \psi_{rII}(x_r)
\]

Así:
\[
C_2 = A
\]

Las derivadas de las funciones de onda en I y en II también deben tender al
mismo valor en $x_r=0$:
\[
\lim_{\scriptsize x_r\to\,0} \psi'_{rI}(x_r)
=
\lim_{\scriptsize x_r\to\,0} \psi'_{rII}(x_r)
\]

Calculemos las derivadas de las funciones de onda en las zonas I y II:
\begin{align*}
\psi'_{rI} &= \frac{d\psi_{rI}(x_r)}{dx_r}
  = C_2\,\sqrt{2(1-\epsilon_r)}\,e^{\sqrt{2(1-\epsilon_r)}\,x_r}
  = A\,\sqrt{2(1-\epsilon_r)}\,e^{\sqrt{2(1-\epsilon_r)}\,x_r}\\
\psi'_{rII} &= \frac{d\psi_{rII}(x_r)}{dx_r}
  = -A\,\sqrt{2\epsilon_r}\,\sen\left(\sqrt{2\epsilon_r}\,x_r\right)
  + B\,\sqrt{2\epsilon_r}\,\cos\left(\sqrt{2\epsilon_r}\,x_r\right)
\end{align*}

Luego:
\[
A\,\sqrt{2(1-\epsilon_r)} =  B\,\sqrt{2\epsilon_r}
\]
\[
B = \sqrt{\frac{1-\epsilon_r}{\epsilon_r}}\,A
\]

Las funciones de onda en las zonas II y III deben tender al mismo valor en
en el otro extremo del pozo de potencial, $x_r=l_r$:
\[
\lim_{\scriptsize x_r\to\,l_r} \psi_{rII}(x_r)
=
\lim_{\scriptsize x_r\to\,l_r} \psi_{rIII}(x_r)
\]

\[
A\,\cos\left(l_r\sqrt{2\epsilon_r}\right)
 + A\,\sqrt{\frac{1-\epsilon_r}{\epsilon_r}}
   \,\sen\left(l_r\,\sqrt{2\epsilon_r}\right)
=
C_5\,\,e^{-l_r\sqrt{2(1-\epsilon_r}}
\]

Despejando $C_5$:
\begin{equation}
\label{c5_1}
C_5 = A\,e^{l_r\sqrt{2(1-\epsilon_r)}}
  \left\lbrace\cos\left(l_r\sqrt{2\epsilon_r}\right) 
  + \sqrt{\frac{1-\epsilon_r}{\epsilon_r}}\,
         \sen\left(l_r\sqrt{2\epsilon_r}\right)\right\rbrace
\end{equation}

\subsection{Ecuación de valores propios}
Las derivadas de la función de onda en las zonas II y III deben coincidir en
$x_r=l_r$:
\[
\lim_{\scriptsize x_r\to\,l_r} \psi'_{rII}(x_r)
=
\lim_{\scriptsize x_r\to\,l_r} \psi'_{rIII}(x_r)
\]

La derivada de $\psi_{rIII}$ es:
\[
\psi'_{rIII}(x_r) = \frac{d\psi_{rIII}(x_r)}{dx_r}
  = -C_5\,\sqrt{2(1-\epsilon_r)}\,e^{-\sqrt{2(1-\epsilon_r)}\,x_r}
\]

Entonces:
\[
-A\,\sqrt{2\epsilon_r}\,\sen\left(l_r\sqrt{2\epsilon_r}\right)
 + A\,\sqrt{\frac{1-\epsilon_r}{\epsilon_r}}\,\sqrt{2\epsilon_r}
   \,\cos\left(l_r\,\sqrt{2\epsilon_r}\right)
=
-C_5\,\sqrt{2(1-\epsilon_r)}\,e^{-l_r\sqrt{2(1-\epsilon_r)}}
\]

Despejando $C_5$:
\begin{equation}
\label{c5_2}
C_5 = A\,e^{l_r\sqrt{2(1-\epsilon_r)}}\,\sqrt{\frac{\epsilon_r}{1-\epsilon_r}}
  \left\lbrace\sen\left(l_r\sqrt{2\epsilon_r}\right)\,
  - \sqrt{\frac{1-\epsilon_r}{\epsilon_r}}\,
       \cos\left(l_r\sqrt{2\epsilon_r}\right)\right\rbrace
\end{equation}

Las expresiones \ref{c5_1} y \ref{c5_2} son iguales. Simplificando:
\[
\cos\left(l_r\sqrt{2\epsilon_r}\right) 
  + \sqrt{\frac{1-\epsilon_r}{\epsilon_r}}\,
         \sen\left(l_r\sqrt{2\epsilon_r}\right)
=
\sqrt{\frac{\epsilon_r}{1-\epsilon_r}}
  \left\lbrace\sen\left(l_r\sqrt{2\epsilon_r}\right)\,
  - \sqrt{\frac{1-\epsilon_r}{\epsilon_r}}\,
       \cos\left(l_r\sqrt{2\epsilon_r}\right)\right\rbrace
\]

Pasando a la izquierda la raíz cuadrada que está junto al signo igual:
\[
\sqrt{\frac{1-\epsilon_r}{\epsilon_r}}
\cos\left(l_r\sqrt{2\epsilon_r}\right)\,
  + \frac{1-\epsilon_r}{\epsilon_r}\,
         \sen\left(l_r\sqrt{2\epsilon_r}\right)
=
  \sen\left(l_r\sqrt{2\epsilon_r}\right)\,
  - \sqrt{\frac{1-\epsilon_r}{\epsilon_r}}\,
       \cos\left(l_r\sqrt{2\epsilon_r}\right)
\]

Pasando los cosenos a la izquierda y los senos a la derecha:
\[
2\sqrt{\frac{1-\epsilon_r}{\epsilon_r}}
\cos\left(l_r\sqrt{2\epsilon_r}\right) 
=
  \sen\left(l_r\sqrt{2\epsilon_r}\right)\,
  - \frac{1-\epsilon_r}{\epsilon_r}\,
       \sen\left(l_r\sqrt{2\epsilon_r}\right)
\]

\[
2\sqrt{\frac{1-\epsilon_r}{\epsilon_r}}
\cos\left(l_r\sqrt{2\epsilon_r}\right) 
=
  \left[1-\frac{1-\epsilon_r}{\epsilon_r}\right]
  \sen\left(l_r\sqrt{2\epsilon_r}\right)
\]

\[
2\sqrt{\frac{1-\epsilon_r}{\epsilon_r}}
\cos\left(l_r\sqrt{2\epsilon_r}\right) 
=
  \frac{2\epsilon_r-1}{\epsilon_r}
  \sen\left(l_r\sqrt{2\epsilon_r}\right)
\]

La ecuación de valores propios es:
\begin{equation}
\label{eigenvalue_eq}
\tan\left(l_r\sqrt{2\epsilon_r}\right)
=
\frac{2\epsilon_r}{2\epsilon_r-1}
\sqrt{\frac{1-\epsilon_r}{\epsilon_r}}
\end{equation}

Esta ecuación no se puede resolver exactamente, sino que debemos utilizar
métodos numéricos.

\subsection{Funciones propias}
Resumiendo y teniendo en cuenta los posibles valores propios, $\epsilon_r$,
la función de onda en cada zona es:
\begin{equation}
\label{wavefunctions}
\psi_r(x_r)=
\begin{cases}
A\,e^{\sqrt{2(1-\epsilon_r)}\,x_r}
& -\infty<x_r<0\\
A\,\left[\cos\left(\sqrt{2\epsilon_r}\,x_r\right)
  + \sqrt{\frac{1-\epsilon_r}{\epsilon}}\,
    \sen\left(\sqrt{2\epsilon_r}\,x_r\right)\right]
& 0<x_r<l_r\phantom{\rule[-9pt]{0em}{0em}}\\
A\,\left[\cos\left(l_r\sqrt{2\epsilon_r}\right)
  + \sqrt{\frac{1-\epsilon_r}{\epsilon}}\,
    \sen\left(l_r\sqrt{2\epsilon_r}\right)\right]
    e^{-\sqrt{2(1-\epsilon_r)}\,(x_r-l_r)}
& l_r<x_r<+\infty
\end{cases}
\end{equation}

\subsection{Cuadrados de la función de onda en cada zona}
Necesitamos calcular el cuadrado de la función de onda pues tenemos
que normalizarla:

\subsubsection{Zone I}
\[
\left|\psi_{rI}(x_r)\right|^2
= A^2\left(e^{\sqrt{2(1-\epsilon_r)}\,x_r}\right)^2
= A^2\,e^{2\sqrt{2(1-\epsilon_r)}\,x_r}
\]

\subsubsection{Zone II}
\begin{align*}
\left|\psi_{rII}(x_r)\right|^2
&= A^2\,\left\lbrace\cos\left(\sqrt{2\epsilon_r}\,x_r\right)
  + \sqrt{\frac{1-\epsilon_r}{\epsilon_r}}\,
    \sen\left(\sqrt{2\epsilon_r}\,x_r\right)
\right\rbrace^2\\
&= A^2\,\left\lbrace
 \cos^2\left(\sqrt{2\epsilon_r}\,x_r\right)\,
 + \frac{1-\epsilon_r}{\epsilon_r}\,\sen^2\left(\sqrt{2\epsilon_r}\,x_r\right)
 + 2\sqrt{\frac{1-\epsilon_r}{\epsilon_r}}\,
 \sen\left(\sqrt{2\epsilon_r}\,x_r\right)\,
 \cos\left(\sqrt{2\epsilon_r}\,x_r\right)
\right\rbrace\\
&= A^2\,\left\lbrace
 1 - \sen^2\left(\sqrt{2\epsilon_r}\,x_r\right)\,
 + \frac{1-\epsilon_r}{\epsilon_r}\,\sen^2\left(\sqrt{2\epsilon_r}\,x_r\right)
 + \sqrt{\frac{1-\epsilon_r}{\epsilon_r}}\,
 \sen\left(2\sqrt{2\epsilon_r}\,x_r\right)
\right\rbrace\\
&= A^2\,\left\lbrace
 1  
 + \left(-1+\frac{1-\epsilon_r}{\epsilon_r}\right)\,\sen^2\left(\sqrt{2\epsilon_r}\,x_r\right)
 + \sqrt{\frac{1-\epsilon_r}{\epsilon_r}}\,
 \sen\left(2\sqrt{2\epsilon_r}\,x_r\right)
\right\rbrace\\
&= A^2\,\left\lbrace
 1  
 + \frac{1-2\epsilon_r}{\epsilon_r}\,\sen^2\left(\sqrt{2\epsilon_r}\,x_r\right)
 + \sqrt{\frac{1-\epsilon_r}{\epsilon_r}}\,
 \sen\left(2\sqrt{2\epsilon_r}\,x_r\right)
\right\rbrace
\end{align*}

\subsubsection{Zone III}

\begin{align*}
\left|\psi_{rIII}(x_r)\right|^2
&= A^2\,\left\lbrace\cos\left(l_r\sqrt{2\epsilon_r}\right)
  + \sqrt{\frac{1-\epsilon_r}{\epsilon_r}}\,
    \sen\left(l_r\sqrt{2\epsilon_r}\right)
\right\rbrace^2
\left\lbrace
\,e^{-\sqrt{2(1-\epsilon_r)}\,(x_r-l_r)}
\right\rbrace^2\\
&= A^2\,\left\lbrace
 \cos^2\left(l_r\sqrt{2\epsilon_r}\right)\,
 + \frac{1-\epsilon_r}{\epsilon_r}\,\sen^2\left(l_r\sqrt{2\epsilon_r}\right)
 + \sqrt{\frac{1-\epsilon_r}{\epsilon_r}}\,
 \sen\left(2l_r\sqrt{2\epsilon_r}\right)
\right\rbrace
\,e^{-2\sqrt{2(1-\epsilon_r)}\,(x_r-l_r)}\\
&= A^2\,\left\lbrace
 1 - \sen^2\left(l_r\sqrt{2\epsilon_r}\right)\,
 + \frac{1-\epsilon_r}{\epsilon_r}\,\sen^2\left(l_r\sqrt{2\epsilon_r}\right)
 + \sqrt{\frac{1-\epsilon_r}{\epsilon_r}}\,
 \sen\left(2l_r\sqrt{2\epsilon_r}\right)
\right\rbrace
\,e^{-2\sqrt{2(1-\epsilon_r)}\,(x_r-l_r)}\\
&= A^2\,\left\lbrace
 1 
 + \left(-1+\frac{1-\epsilon_r}{\epsilon_r}\right)\,\sen^2\left(l_r\sqrt{2\epsilon_r}\right)
 + \sqrt{\frac{1-\epsilon_r}{\epsilon_r}}\,
 \sen\left(2l_r\sqrt{2\epsilon_r}\right)
\right\rbrace
\,e^{-2\sqrt{2(1-\epsilon_r)}\,(x_r-l_r)}\\
&= A^2\,\left\lbrace
 1 
 + \frac{1-2\epsilon_r}{\epsilon_r}\,\sen^2\left(l_r\sqrt{2\epsilon_r}\right)
 + \sqrt{\frac{1-\epsilon_r}{\epsilon_r}}\,
 \sen\left(2l_r\sqrt{2\epsilon_r}\right)
\right\rbrace
\,e^{-2\sqrt{2(1-\epsilon_r)}\,(x_r-l_r)}\\
&= A^2\,e^{-2\sqrt{2(1-\epsilon_r)}\,(x_r-l_r)}
\end{align*}

Con respecto al último paso, nótese que:
\[
 \frac{1-2\epsilon_r}{\epsilon_r}\,\sen^2\left(l_r\sqrt{2\epsilon_r}\right)
 + \sqrt{\frac{1-\epsilon_r}{\epsilon_r}}\,
 \sen\left(2l_r\sqrt{2\epsilon_r}\right) = 0
\]
sólo si $\epsilon_r$ es un valor propio.  

Para demostrar esto, supondremos que $\epsilon_r$ es un valor propio, entonces
debe cumplir la ecuación de valores propios, \ref{eigenvalue_eq}:
\[
\tan\left(l_r\sqrt{2\epsilon_r}\right)
=
\frac{2\epsilon_r}{2\epsilon_r-1}
\sqrt{\frac{1-\epsilon_r}{\epsilon_r}}
\]

\[
\frac{\sen\left(l_r\sqrt{2\epsilon_r}\right)}
{\cos\left(l_r\sqrt{2\epsilon_r}\right)}
=
\frac{2\epsilon_r}{2\epsilon_r-1}
\sqrt{\frac{1-\epsilon_r}{\epsilon_r}}
\]

Reagrupando términos:
\[
\frac{2\epsilon_r-1}{\epsilon_r}\,
\sen\left(l_r\sqrt{2\epsilon_r}\right)
=
2\sqrt{\frac{1-\epsilon_r}{\epsilon_r}}
\cos\left(l_r\sqrt{2\epsilon_r}\right)
\]

\[
\frac{1-2\epsilon_r}{\epsilon_r}\,
\sen\left(l_r\sqrt{2\epsilon_r}\right)
+
2\sqrt{\frac{1-\epsilon_r}{\epsilon_r}}
\cos\left(l_r\sqrt{2\epsilon_r}\right)
=0
\]

Multiplicando ambos miembros por el seno del ángulo:
\[
\frac{1-2\epsilon_r}{\epsilon_r}\,
\sen^2\left(l_r\sqrt{2\epsilon_r}\right)
+
2\sqrt{\frac{1-\epsilon_r}{\epsilon_r}}
\sen\left(l_r\sqrt{2\epsilon_r}\right)\,
\cos\left(l_r\sqrt{2\epsilon_r}\right)
=0
\]

Entonces hemos demostrado que:
\[
\frac{1-2\epsilon_r}{\epsilon_r}\,
\sen^2\left(l_r\sqrt{2\epsilon_r}\right)
+
\sqrt{\frac{1-\epsilon_r}{\epsilon_r}}
\sen\left(2l_r\sqrt{2\epsilon_r}\right)
=0
\]

\subsection{Normalización}
La función de onda se tiene que normalizar. Esto es:
\[
\int_{-\infty}^{+\infty} \left|\psi_r(x_r)\right|^2 dx_r = 1
\]

\[
\int_{-\infty}^{0}\left|\psi_{rI}(x_r)\right|^2 dx_r 
 + \int_{0}^{l_r}\left|\psi_{rII}(x_r)\right|^2 dx_r 
 + \int_{l_r}^{+\infty}\left|\psi_{rIII}(x_r)\right|^2 dx_r 
 = 1
\]

Debemos calcular tres integrales:
\begin{equation}
\label{normalization_eq}
I_1 + I_2 + I_3 = 1
\end{equation}

\subsubsection{Primera integral: $I_1$}

\[
I_1 = \int_{-\infty}^{0}\left|\psi_{rI}(x_r)\right|^2 dx_r 
= 
A^2\,\int_{-\infty}^0 e^{2\sqrt{2(1-\epsilon_r)}\,x_r} dx_r
\]

Cambio de variable:
\[
2\sqrt{2(1-\epsilon_r)}\,x_r = t_r
\hspace{2em}
dx_r = \frac{1}{2\sqrt{2(1-\epsilon_r)}}\,dt_r
\]

\[
I_1
= A^2\,\frac{1}{2\sqrt{2(1-\epsilon_r)}}\int_{-\infty}^0 e^{t_r} dt_r
= A^2\,\frac{1}{2\sqrt{2(1-\epsilon_r)}}(e^0-e^{-\infty})
\]

Luego:
\begin{equation}
\label{I1}
I_1 = \frac{A^2}{2\sqrt{2(1-\epsilon_r)}}
\end{equation}

\subsubsection{Tercera integral: $I_3$}

Nos saltamos la segunda integral por el momento:
\[
I_3 = \int_{l_r}^{+\infty}\left|\psi_{rIII}(x_r)\right|^2 dx_r 
= 
A^2\,\int_{l_r}^{+\infty} e^{-2\sqrt{2(1-\epsilon_r)}\,(x_r-l_r)} dx_r
\]

Cambio de variable:
\[
-2\sqrt{2(1-\epsilon_r)}\,(x_r-l_r) = t_r
\hspace{2em}
dx_r = \frac{-1}{2\sqrt{2(1-\epsilon_r)}}\,dt_r
\]

\[
I_3
= \frac{-A^2}{2\sqrt{2(1-\epsilon_r)}}\int_{0}^{-\infty} e^{t_r} dt_r
= \frac{-A^2}{2\sqrt{2(1-\epsilon_r)}}(e^{-\infty}-e^{0})
\]

Entonces:
\begin{equation}
\label{I3}
I_3 = \frac{A^2}{2\sqrt{2(1-\epsilon_r)}}
\end{equation}

\subsubsection{Segunda integral: $I_2$}

Como veremos, esta integral se resolverá en tres partes,
$I_{21}$, $I_{22}$, $I_{23}$:
\[
I_2 = \int_{0}^{l_r}\left|\psi_{rII}(x_r)\right|^2 dx_r 
= 
A^2\,\int_{0}^{l_r}
\left\lbrace
1
 + \frac{1-2\epsilon_r}{\epsilon_r}\,\sen^2\left(\sqrt{2\epsilon_r}\,x_r\right)
 + \sqrt{\frac{1-\epsilon_r}{\epsilon_r}}\,\sen\left(2\sqrt{2\epsilon_r}\,x_r\right)
\right\rbrace
dx_r
\]

\[
I_2 = A^2\,\left\lbrace I_{21} + I_{22} + I_{23}\right\rbrace
\]

\subsubsection{Segunda integral, primera parte: $I_{21}$}

\[
I_{21}
= 
\int_{0}^{l_r} dx_r
\]

\begin{equation}
\label{I21}
I_{21} = l_r
\end{equation}

\subsubsection{Segunda integral, segunda parte: $I_{22}$}

\[
I_{22}
= 
\frac{1-2\epsilon_r}{\epsilon_r}\,
\int_{0}^{l_r}\sen^2\left(\sqrt{2\epsilon_r}\,x_r\right) dx_r
\]

Cambio de variable:
\[
\sqrt{2\epsilon_r}\,x_r = t_r
\hspace{2em}
dx_r = \frac{1}{\sqrt{2\epsilon_r}}\,dt_r
\]

\[
I_{22}
= 
\frac{1-2\epsilon_r}{\epsilon_r}\,
\frac{1}{\sqrt{2\epsilon_r}}\,
\int_{0}^{l_r\sqrt{2\epsilon_r}}\hspace*{-2em}\sen^2 t_r\, dt_r
\]

Escribamos $\sen^2 t_r$ de una forma diferente:
\[
\cos 2t_r = \cos^2 t_r - \sen^2 t_r = 1-\sen^2 t_r - \sen^2 t_r
= 1-2\sen^2 t_r
\]

\[
\sen^2 t_r = \frac{1}{2} - \frac{1}{2}\,\cos (2t_r)
\]

\[
I_{22}
= 
\frac{1-2\epsilon_r}{\epsilon_r\sqrt{2\epsilon_r}}\,
\left\lbrace
\frac{1}{2}\int_{0}^{l_r\sqrt{2\epsilon_r}}\hspace*{-2em}\,dt_r
- \frac{1}{2}\int_{0}^{l_r\sqrt{2\epsilon_r}}\hspace*{-2em}\cos(2t_r)\, dt_r
\right\rbrace
\]

Cambio de variable:
\[
2t_r = z_r
\hspace{2em}
dt_r = \frac{1}{2}\,dz_r
\]

\begin{align*}
I_{22}
&= 
\frac{1-2\epsilon_r}{\epsilon_r\sqrt{2\epsilon_r}}\,
\left\lbrace
\frac{l_r\sqrt{2\epsilon_r}}{2}
- \frac{1}{4}\int_{0}^{2l_r\sqrt{2\epsilon_r}}\hspace*{-2em}\cos(2t_r)\, dt_r
\right\rbrace\\
&= 
\frac{1-2\epsilon_r}{\epsilon_r\sqrt{2\epsilon_r}}\,
\left\lbrace
\frac{2l_r\sqrt{2\epsilon_r}}{4}
- \frac{\sen\left(2l_r\sqrt{2\epsilon_r}\right)}{4}
\right\rbrace
\end{align*}

\begin{equation}
\label{I22}
I_{22} =
\frac{1-2\epsilon_r}{4\epsilon_r\sqrt{2\epsilon_r}}\,
\left\lbrace
2l_r\sqrt{2\epsilon_r}
- \sen\left(2l_r\sqrt{2\epsilon_r}\right)
\right\rbrace
\end{equation}

\subsubsection{Segunda integral, tercera parte: $I_{23}$}

\[
I_{23}
= 
\sqrt{\frac{1-\epsilon_r}{\epsilon_r}}\,
\int_{0}^{l_r}\sen\left(2\sqrt{2\epsilon_r}\,x_r\right)\, dx_r
\]

Cambio de variable:
\[
2\sqrt{2\epsilon_r}\,x_r = t_r
\hspace{2em}
dx_r = \frac{1}{2\sqrt{2\epsilon_r}}\,dt_r
\]

\begin{align*}
I_{23}
&= 
\sqrt{\frac{1-\epsilon_r}{\epsilon_r}}\,
\frac{1}{2\sqrt{2\epsilon_r}}\,
\int_{0}^{2l_r\sqrt{2\epsilon_r}}\hspace*{-2em}\sen t_r\, dt_r\\
&= 
-\sqrt{\frac{1-\epsilon_r}{\epsilon_r}}\,
\frac{1}{2\sqrt{2\epsilon_r}}\,
\left[\cos\left(2l_r\sqrt{2\epsilon_r}\right) - cos 0\right]
\end{align*}

\begin{equation}
\label{I23}
I_{23} =
\sqrt{\frac{1-\epsilon_r}{\epsilon_r}}\,
\frac{1}{2\sqrt{2\epsilon_r}}\,
\left[1 - \cos\left(2l_r\sqrt{2\epsilon_r}\right)\right]
\end{equation}

So $I_2$ is:
\begin{equation}
\label{I2}
I_2 
= A^2\,
\left\lbrace
l_r
+
\frac{1-2\epsilon_r}{4\epsilon_r\sqrt{2\epsilon_r}}\,
\left[
2l_r\sqrt{2\epsilon_r}
- \sen\left(2l_r\sqrt{2\epsilon_r}\right)
\right]
+
\sqrt{\frac{1-\epsilon_r}{\epsilon_r}}\,
\frac{1}{2\sqrt{2\epsilon_r}}\,
\left[1 - \cos\left(2l_r\sqrt{2\epsilon_r}\right)\right]
\right\rbrace
\end{equation}

De la condición de normalización, \ref{normalization_eq}, podemos obtener el
valor de $A$:
\[
A = \frac{1}{\sqrt{I_1 + I_2 + I_3}}
\]

