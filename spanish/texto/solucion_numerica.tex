% texto/solucion_numerica.tex

\chapter{Solución numérica de los estados ligados}

\section{Ecuación de valores propios}

La ecuación de valores propios, \ref{eigenvalue_eq}, tiene dos partes:

\[
\tg\left(l_r\sqrt{2\epsilon_r}\right)
=
\frac{2\epsilon_r}{2\epsilon_r-1}
\sqrt{\frac{1-\epsilon_r}{\epsilon_r}}
\]

\[
f_1(\epsilon_r)
=
f_2(\epsilon_r)
\]

Una es la función tangente:
\begin{equation}
\label{eigenval_func1}
f_1(\epsilon_r) = \tg\left({l_r\sqrt{2\epsilon_r}}\right)
\end{equation}

y la otra es una función no trigonométrica:
\begin{equation}
\label{eigenval_func2}
f_2(\epsilon_r) = \frac{2\epsilon_r}{2\epsilon_r-1}
\sqrt{\frac{1-\epsilon_r}{\epsilon_r}}
\end{equation}

\subsection{Función tangente}

\subsubsection{Función: $\tg\theta$}
La función tangente, $y = \tg\theta$ no es continua; tiene infinitas
asíntotas verticales:
\[
y = \tg\theta = \frac{\sen\theta}{\cos\theta}
\]

Estas asíntotas corresponden a los puntos del eje $x$ en los que $\cos\theta=0$.
Estos son los $\theta=(2n-1)\pi/2$, donde $n=\pm 1, \pm 2, \cdots$.
La función tangente por esta razón, discontinua; pero entre las asíntotas es
continua.

Ahora veamos los ceros de la función tangente. La tangente será cero cuando
el seno del ángulo sea cero. Esto ocurre en los puntos:
\begin{equation}
\label{asymptotes_tan}
\theta=\cdots,-3\pi,-2\pi,-\pi,0,\pi,2\pi,3\pi,\cdots
\end{equation}

Analicemos si es creciente o decreciente entre las asíntotas. Si la primera
derivada es positiva la función es creciente y decreciente si es negativa.
La primera derivada es creciente pues el cuadrado de un número real siempre
es positivo:
\[
y' = \frac{d(\tg\theta)}{d\theta} = \frac{1}{\cos^{2}\theta}
\]

La segunda derivada es:
\[
y'' = \frac{d(1/\cos^{2}\theta)}{d\theta} = 
\frac{-\sen(2\theta)}{\cos^{4}\theta}
\]

La tangente será cóncava si la segunda derivada es positiva y convexa
si esta segunda derivada fuera negativa.

Razonemos el signo. Por un lado, $cos^{4}\theta$ es siempre positivo
(potencia par).
La convexidad o concavidad dependerá del signo de $-\sen (2\theta)$.
Veamos:

Como la función tangente es periódica, vamos a razonar la concavidad y la
convexidad entre dos asíntotas consecutivas; lo que se deduzca se puede
generalizar al resto de intervalos.

La función $\sen\theta$ es negativa en el intervalo $(-\pi,0)$ y positiva en
$(0,\pi)$.

Se extender el razonamiento a la función $\sen (2\theta)$ teniendo en cuenta
que el dominio se divide entre dos, lo que significa que los valores de
$\theta$ anteriores se deben dividir entre dos.
Resumiendo, $\sen (2\theta)$ será negativa en el intervalo abierto $(-\pi/2,0)$
y positiva en $(0,\pi/2)$.

A $-\sen (2\theta)$ le ocurrirá lo contrario debido al signo negativo:
será positiva en el intervalo $(-\pi/2,0)$ y negativa en $(0,\pi/2)$.

Como $(-\pi/2,+\pi/2)$ forman un intervalo entre dos asíntotas consecutivas,
la primera mitad de éste es cóncavo y la segunda mitad convexo.

\begin{figure}[ht]
\centering
\begin{tikzpicture}
\begin{axis}[
restrict y to domain=-10:10,
samples=2000,
%width=7cm, height=5cm,
xmin=-3.4*pi/2-0.1, xmax=3.3*pi/2+0.1,
xtick={-4.7124,-1.5708,...,10},
xticklabels={$-\frac32 \pi$,$-\pi/2$,$\pi/2$,$\frac32 \pi$},
ytick={0},
yticklabels={0},
axis x line=center,
axis y line=center,
minor tick num=1,
use background
]

% Asíntotas
\foreach \i in {-3*pi/2,-pi/2,pi/2,3*pi/2}
{\addplot[gray!50,dashed] coordinates{(\i,-10) (\i,10)};};%

% Función tangente
\addplot[blue] gnuplot[id=tangens,domain=-3.4*pi/2:3.4*pi] {tan(x)};

\legend{$y=\tg(\theta)$}
\end{axis}
\end{tikzpicture}
\caption{La función tangente, $y=\tg{\theta}$ es periódica}
\end{figure}

\subsubsection{Función: $\tg\left(l_r\sqrt{2\epsilon_r}\right)$}
Muchas características generales estudiadas para $\tg\theta$ siguen siendo
válidas ahora, pero hay unos cambios:

El dominio ya no es el de todos los reales menos las posiciones donde hay
asíntotas; ahora la energía reducida debe estar comprendida entre cero y uno,
$0<\epsilon_r<1$, lo que limita el número de asíntotas que habrá.
Recordando los ángulos en los que la función tangente no está definida,
\ref{asymptotes_tan}, y descartando los valores negativos:

\[
\theta=l_r\sqrt{2\epsilon_r}=0,\pi,2\pi,3\pi,\cdots
\]

\[
l_r\sqrt{2\epsilon_r}=(2n-1)\,\frac{\pi}{2}
;\hspace*{1em}
n=1,2,3,\cdots
\]

El entero $n$ tiene un límite superior, el correspondiente al valor
$\epsilon_r=1$, que es inalcanzable:
\begin{equation}
\label{n_limsup}
n < \frac{\sqrt{2}}{\pi}\,l_r + \frac{1}{2}
\end{equation}

Por otra parte, este valor de n máximo posible nos da el número de valores
propios o de estados ligados.

Además no puede valer cero; su menor valor es uno: $n\ge 1$. Están
prohibidos los valores $n<1$. En este caso, para un infinitésimo $\delta$:
\[
1\le n\le\left\lfloor \frac{\sqrt{2}}{\pi}\,l_r+\frac{1}{2}-\delta\right\rfloor
\]

De aquí que:
\[
1 \le \left\lfloor\frac{\sqrt{2}}{\pi}\,l_r+\frac{1}{2}-\delta\right\rfloor
\]

\[
1 \le \frac{\sqrt{2}}{\pi}\,l_r+\frac{1}{2}-\delta
\]

\[
1 < \frac{\sqrt{2}}{\pi}\,l_r+\frac{1}{2}
\]

\[
\frac{1}{2} < \frac{\sqrt{2}}{\pi}\,l_r
\]

Así, para que haya estados ligados, $l_r$ debe ser superior a:
\[
l_r > \frac{\pi}{2\sqrt{2}} \approx 1,11072073\cdots
\]

Como la variable independiente de la función, $l_r\sqrt{2\epsilon_r}$,
es una función creciente de $\epsilon_r$, las asíntotas ya no están
equiespaciadas, sino que esta distancia aumenta (hasta que se termina el
dominio de la función con $\epsilon_r=1$). Por ejemplo, para un
valor $l_r=11,45575$ se puede ver una representación en la figura
~\ref{fig:eigenvalfuncs12}.

\subsection{Función no trigonométrica}
La función no trigonométrica, \ref{eigenval_func2}:
\[
\frac{2\epsilon_r}{2\epsilon_r-1}
\sqrt{\frac{1-\epsilon_r}{\epsilon_r}}
\]
está representada en la figura ~\ref{fig:eigenvalfuncs12}.

Su dominio consiste en todos los reales comprendidos entre 0 y 1, excepto
$\epsilon_r=0.5$, que es la asíntota vertical de esta función.  

\begin{figure}[ht]
\centering
\begin{tikzpicture}
\begin{groupplot}[
group style={
  group name=mygroup,
  group size=2 by 1,
  xlabels at=edge bottom,
  ylabels at=edge left
},
footnotesize
]

\nextgroupplot[
restrict y to domain=-10:10,
samples=3000,
xmin=0,xmax=1,
ymin=-10,ymax=10,
width=7cm,
xlabel=$\epsilon_r$,
ytick={0},
yticklabels={0},
minor tick num=1,
use background
]
% Función tangente
\addplot[blue] gnuplot[id=tangenvar,domain=0:1] {tan(11.45575*sqrt(2*x))};

% Asíntotas
\foreach \i in {pi/2,pi/2,3*pi/2,5*pi/2,7*pi/2,9*pi/2}
{\addplot[gray!50,dashed] coordinates{(0.5*(\i/11.45575)^2,-10)(0.5*(\i/11.45575)^2,10)};};%
\legend{$f_1(\epsilon_r) =\tg(l_r\sqrt{2\epsilon_r)}$}

\nextgroupplot[
restrict y to domain=-50:50,
samples=2000,
width=7cm,
%width=7cm, height=5cm,
xmin=0, xmax=1.0,
ymin=-30,ymax=30,
ytick={0},
yticklabels={0},
%axis x line=center,
%axis y line=center,
xlabel=$\epsilon_r$,
minor tick num=1,
use background
]

% Función tangente
\addplot[red] gnuplot[id=tangenvar,domain=0:1] {2*x*sqrt((1-x)/x)/(2*x-1)};

\legend{
$f_2(\epsilon_r) = \frac{2\epsilon_r}{2\epsilon_r-1}\sqrt{\frac{1-\epsilon_r}{\epsilon_r}}$
}

% Asíntotas
\addplot[gray!50,dashed] coordinates{(0,-50) (0,50)};
\addplot[gray!50,dashed] coordinates{(0.5,-50) (0.5,50)};

\end{groupplot}
\end{tikzpicture}

\caption{Las dos funciones que forman la ecuación de valores propios}
\label{fig:eigenvalfuncs12}
\end{figure}

Los valores propios se encuentran en los puntos de corte de estas dos
funciones, ver figura ~\ref{fig:eigenvalues}:

\begin{figure}[ht]
\centering
\begin{tikzpicture}
[scale=0.5]
\begin{axis}[
restrict y to domain=-10:10,
samples=2000,
width=12cm,
xmin=0, xmax=1,
ymin=-10,ymax=10,
ytick={0},
yticklabels={0},
minor tick num=1,
%axis x line=center,
%axis y line=center,
xlabel=$\epsilon_r$,
use background,
]
% Función tangente
\addplot[blue] gnuplot[id=tangenvar,domain=0:1] {tan(11.45575*sqrt(2*x))};
% Función otra
\addplot[red] gnuplot[id=tangen,domain=0:1] {2*x*sqrt((1-x)/x)/(2*x-1)};
\legend{$\tg(l_r\sqrt{2\epsilon_r)}$,
  $\frac{2\epsilon_r}{2\epsilon_r-1}\sqrt{\frac{1-\epsilon_r}{\epsilon_r}}$}

% Asíntotas
\foreach \i in {0,0.5}
{\addplot[gray!50,dashed] coordinates{(\i,-10) (\i,10)};};%
\foreach \i in {pi/2,3*pi/2,5*pi/2,7*pi/2,9*pi/2}
{\addplot[gray!50,dashed] coordinates{(0.5*(\i/11.45575)^2,-10)(0.5*(\i/11.45575)^2,10)};};%

\end{axis}
\end{tikzpicture}
\caption{Los puntos de corte representan los valores propios $\epsilon_r$}
\label{fig:eigenvalues}
\end{figure}

\section{Funciones propias}
Las funciones de ondas (una por cada valor propio) en las distintas zonas se
calculó en ~\ref{wavefunctions}:
\[
\psi_r(x_r)=
\begin{cases}
A\,e^{\sqrt{2(1-\epsilon_r)}\,x_r}
& -\infty<x_r<0\\
A\,\left[\cos\left(\sqrt{2\epsilon_r}\,x_r\right)
  + \sqrt{\frac{1-\epsilon_r}{\epsilon}}\,
    \sen\left(\sqrt{2\epsilon_r}\,x_r\right)\right]
& 0<x_r<l_r\phantom{\rule[-10pt]{0em}{0em}}\\
A\,\left[\cos\left(l_r\sqrt{2\epsilon_r}\right)
  + \sqrt{\frac{1-\epsilon_r}{\epsilon}}\,
    \sen\left(l_r\sqrt{2\epsilon_r}\right)\right]
    e^{-\sqrt{2(1-\epsilon_r)}\,(x_r-l_r)}
& l_r<x_r<+\infty
\end{cases}
\]

donde $A$ se calcula:
\[
A = \frac{1}{\sqrt{I_1 + I_2 + I_3}}
\]

y las constantes $I_i$ ($i=1,2,3$):
\begin{align*}
&I_1 = \frac{1}{2\sqrt{2(1-\epsilon_r)}}\\
&I_2 
= 
\left\lbrace
l_r
+
\frac{1-2\epsilon_r}{4\epsilon_r\sqrt{2\epsilon_r}}\,
\left[
2l_r\sqrt{2\epsilon_r}
- \sen\left(2l_r\sqrt{2\epsilon_r}\right)
\right]
+
\frac{1 - \cos\left(2l_r\sqrt{2\epsilon_r}\right)}
{2\sqrt{2\epsilon_r}}\,
\sqrt{\frac{1-\epsilon_r}{\epsilon_r}}
\right\rbrace\\
&I_3 = \frac{1}{2\sqrt{2(1-\epsilon_r)}}
\end{align*}

