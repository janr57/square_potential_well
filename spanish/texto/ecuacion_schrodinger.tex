% texto/ecuacion_schrodinger.tex

\chapter[Ecuación de Schrödinger para el pozo de potencial rectangular]
{Ecuación de Schrödinger unidimensional para el pozo de potencial rectangular}

\section{Pozo de potencial rectangular}
Supondremos que tenemos una partícula de masa $m$, que se encuentra en un
espacio de una dimensión $x$. La energía potencial en todo el espacio vale
$U(x)=U_0$, excepto en una región comprendida en el intervalo $0 < x < l$, donde
$l$ es una coordenada positiva; la energía potencial en esta región vale cero:
$U(x)=0$.

A continuación se presenta una representación del pozo de potencial.

\begin{figure}[ht]
\centering
\begin{tikzpicture}
[
scale=0.75,
%use background
]

\begin{axis}
[xmin=-10, xmax=16, ymin=-4, ymax=3,
xlabel={Posición}, ylabel={Energía},
xtick={0,6}, ytick={-3,0},
xticklabels={0,$l$}, yticklabels={0,$U_0$},
use background % Color definido en la cabecera del documento todo el gráfico
%background color=orange!20, use background, % Color todo el gráfico
%axis background/.style={fill=yellow!20} % Color en el interior de la gráfica
]
\addplot+[sharp plot,no marks,line width=2pt,color=black] coordinates
{(-10,0) (0,0) (0,-3) (6,-3) (6,0) (16,0)};
\end{axis}
\end{tikzpicture}
\caption{Representación del pozo de potencial rectangular}
\end{figure}

Reconocemos tres regiones:
\begin{itemize}
\item Zona I: Posiciones a la izquierda del pozo, $x<0$.
\item Zona II: Posiciones del pozo de potencial, $0<x<l$, donde la
partícula siente una atracción representada por el potencial.
\item Zona III: Posiciones a la derecha del pozo, $x>l$.
\end{itemize}

El pozo de potencial representa la región del eje $x$ donde la partícula
está sujeta a una fuerza atractiva.

\section{Variables adimensionales}
Se podría resolver la ecuación de Schrödinger directamente, utilizando las
variables en el SI, pero vamos a utilizar variables adimensionales porque,
en cierto sentido, los resultados que obtengamos son independientes de
las escalas de masas, longitudes y energías de cada caso particular.
Por supuesto, si las variables son adimensionales también lo será la
ecuación de Scrödinger.

Se utilizará una posición reducida adimensional, $x_r$, definida como:
\begin{equation}
\label{reduced_x}
	x_r = \frac{x}{C}
\end{equation}
$C$ es una constante con dimensiones de longitud, $[C] = L^1$.

De manera similar, necesitamos una energía reducida adimensional, $\epsilon_r$:
\begin{equation}
\label{reduced_e}
	\epsilon_r = \frac{E}{D}
\end{equation}
donde $D$ es una constante con dimensiones de energía, $[D] = ML^2T^{-2}$.

Estas constantes $C$ and $D$ deben depender de magnitudes que aparecen en la
ecuación de Schrödinger en una dimensión, $x$:
\begin{equation}
\label{schro1}
-\frac{\hbar^2}{2m}\frac{d\psi(x)}{dx^2}+U(x)\,\psi(x) = E\,\psi(x)
\end{equation}

Las magnitudes de las que dependerán las constantes anteriores son la
constate de Planck reducida, $\hbar$, la masa de la partícula, $m$, y el
valor del potencial que rodea al pozo, $U_0$.

Sus dimensiones respectivas son acción, masa y energía:
\begin{align*}
&[\hbar] = ML^2T^{-1}\\
&[m] = M\\
&[U_0] = ML^2T^{-2}
\end{align*}

La constante $C$ es una combinación de estas constantes
\[
[C] = \hbar^a \,m^b \,V_0^c =
(ML^2T^{-1})^a \,(M)^b \,(ML^2T^{-2})^c =
M^{a+b+c} \,L^{2a+2c} \,T^{-a-2c}
\]

Los exponentes $a$, $b$ and $c$ se pueden calcular sabiendo que $C$ sólo tiene
dimensiones de longitud ($M^0 L^1 T^0 = L$).
De esta manera se obtiene el siguiente sistema de ecuaciones:
\begin{align*}
&a + b + c = 0\\
&2a+2c = 1\\
&-a-2c = 0
\end{align*}

Las soluciones son: $a=1$ and $b=c=-1/2$. Así, $C$ vale:
\begin{equation}
C = \hbar \,(mU_0)^{-1/2} = \frac{\hbar}{\sqrt{mU_0}}
\end{equation}

El razonamiento del valor de la constante $D$ es más sencillo:
$D=U_0$, puesto que ambas magnitudes, $D$ y $U_0$ tienen dimensiones de
energía.

Así, tenemos:
\begin{equation}
\label{reduced_x2}
x_r = \frac{x}{C}=\frac{x}{\hbar/\sqrt{mU_0}} = \frac{\sqrt{mU_0}}{\hbar}\,x
\end{equation}

y
\begin{equation}
\label{reduced_e2}
\epsilon_r = \frac{E}{D}=\frac{E}{U_0}
\end{equation}

De esta manera, cuando queramos recuperar los valores reales (con las
dimensiones apropiadas), realizaremos los siguientes cálculos:
\[
x = \frac{\hbar}{\sqrt{mU_0}}\,x_r
\]

y
\[
E = U_0\,\epsilon_r
\]

\subsection{Función de onda y ecuación de Schrödinger reducida}
Nótese que $|\psi(x)|^2\,dx$ es una probabilidad, por lo que no tiene
dimensiones. De esto deducimos que la función de onda, $\psi(x)$, debe tener
dimensiones de $L^{-1/2}$.

Aprovechando que la constante $C$ tiene dimensiones de longitud, podemos
definir una función de onda reducida, $\psi_r(x_r)$:
\begin{equation}
\label{reduced_psi}
\psi_r(x_r) = \frac{\psi(x)}{C^{-1/2}}
\end{equation}

Para deducir la ecuación de Schrödinger reducida correspondiente, debemos
partir de la ecuación normal. Transformando ligeramente la \ref{schro1}:
\begin{equation}
\label{schro2}
\frac{\hbar^2}{2m}\frac{d^2\psi(x)}{dx^2} + (E-U(x))\psi(x) = 0
\end{equation}

Se calcula la primera derivada de la función de onda, $d\psi(x)/dx$, 
usando \ref{reduced_psi}:
\[
\frac{d\psi(x)}{dx}
= \frac{d}{dx}\, C^{-1/2}\psi_r(x_r)
= C^{-1/2}\,\frac{d\psi_r(x_r)}{dx}
= C^{-1/2}\,\frac{d\psi_r(x_r)}{dx_r}\,\frac{dx_r}{dx}
\]

De \ref{reduced_x}, podemos calcular $dx_r/dx$, lo que da $C^{-1}$:
\[
\frac{d\psi(x)}{dx}
= C^{-1/2}\,\frac{d\psi_r(x_r)}{dx_r}\,C^{-1}
= C^{-3/2}\,\frac{d\psi_r(x_r)}{dx_r}
\]

Ahora se calcula la segunda derivada, $d^2\psi(x)/dx^2$:
\begin{align*}
\frac{d^2\psi(x)}{dx^2}
&= \frac{d}{dx}\left[\frac{d\psi_r(x_r)}{dx}\right]
= \frac{d}{dx}\left[C^{-3/2}\,\frac{d\psi_r(x_r)}{dx_r}\right]
= C^{-3/2}\,\frac{d}{dx}\,\left[\frac{d\psi_r(x_r)}{dx_r}\right]\\
&= C^{-3/2}\,\frac{d}{dx_r}\,\left[\frac{d\psi_r(x_r)}{dx_r}\right]
	\,\frac{dx_r}{dx}
= C^{-3/2}\,\frac{d^2\psi_r(x_r)}{dx_r^2} \,C^{-1}\\
&= C^{-5/2}\,\frac{d^2\psi_r(x_r)}{dx_r^2}
\end{align*}

\subsection{Ecuación de Schrödinger para las zonas I y III}
En las zonas I y III la energía de potencial es cero.
Substituimos este valor del potencial en la ecuación de Schrödinger,
\ref{schro2},usando la energía reducida, \ref{reduced_e2}, y la función de
onda reducida, \ref{reduced_psi}:
\[
C^{-5/2}\,\frac{d^2\psi_r}{dx_r^2}
	 + \frac{2m}{\hbar^2}\,(\epsilon_r U_0-U_0)\,C^{-1/2}\psi_r=0
\]

Multiplicamos ambos términos por $C^{1/2}$:
\[
C^{-5/2}C^{1/2}\,\frac{d^2\psi_r}{dx_r^2}
	 - \frac{2mU_0}{\hbar^2}\,(1-\epsilon_r)\,C^{-1/2}\,C^{1/2}\,\psi_r=0
\]

\[
C^{-2}\,\frac{d^2\psi_r}{dx_r^2}
	 - \frac{2mU_0}{\hbar^2}\,(1-\epsilon_r)\,\psi_r=0
\]

De \ref{reduced_x2}, podemos deducir que $C^{-2}=mU_0/\hbar^2$:
\[
\frac{mU_0}{\hbar^2}\,\frac{d^2\psi_r}{dx_r^2}
	 - \frac{2mU_0}{\hbar^2}\,(1-\epsilon_r)\,\psi_r=0
\]

La ecuación de estado reducida para las zonas I y III es:
\begin{equation}
\frac{d^2\psi_r}{dx_r^2} + 2\,(1-\epsilon_r)\,\psi_r=0
\end{equation}

\subsection{Ecuación de Schrödinger para la zona II}
En la zona II la energía potencial es cero: $U(x) = 0$.
La ecuación de Schrödinger es:
\[
\label{schro2}
\frac{\hbar^2}{2m}\frac{d^2\psi(x)}{dx^2} + E\,\psi(x) = 0
\]

\[
C^{-5/2}\,\frac{d^2\psi_r}{dx_r^2}
	 + \frac{2m}{\hbar^2}\,\epsilon_r\,U_0\,C^{-1/2}\psi_r=0
\]

Multiplicamos ambos términos por $C^{1/2}$:
\[
C^{-5/2}C^{1/2}\,\frac{d^2\psi_r}{dx_r^2}
	 + \frac{2mU_0}{\hbar^2}\,\epsilon_r\,C^{-1/2}\,C^{1/2}\,\psi_r=0
\]

\[
C^{-2}\,\frac{d^2\psi_r}{dx_r^2}
	 + \frac{2mU_0}{\hbar^2}\,\epsilon_r\,\psi_r=0
\]

De \ref{reduced_x2}, podemos deducir que $C^{-2}=m|U_0|/\hbar^2$:
\[
\frac{mU_0}{\hbar^2}\,\frac{d^2\psi_r}{dx_r^2}
	 + \frac{2mU_0}{\hbar^2}\,\epsilon_r\,\psi_r=0
\]

As we did in the previous subsection, we get the reduced Schrödinger equation:
\begin{equation}
\frac{d^2\psi_r}{dx_r^2} + 2\epsilon_r\,\psi_r=0
\end{equation}

