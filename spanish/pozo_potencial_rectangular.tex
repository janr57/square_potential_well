% pozo_potencial_rectangular.tex
% Copyright (C) 2013 José A. Navarro josea.navarro@murciaeduca.es
\documentclass[svgnames,a4paper,notitlepage]{report}
% Paper dimensions for tikz
\paperwidth=210mm
\paperheight=297mm
\usepackage[spanish,es-noquoting,es-nodecimaldot]{babel}
\usepackage[spanish]{babel}
\usepackage[utf8]{inputenc}
\usepackage[T1]{fontenc}

\usepackage{amsmath}
\usepackage[charter]{mathdesign}
\usepackage{textcomp}
\usepackage[final,babel]{microtype}

\usepackage{siunitx}
%\sisetup{output-decimal-marker = {,}}
\usepackage{xcolor}
\usepackage[pdftex]{graphicx}
\usepackage{svg}
\usepackage{cancel}
\usepackage{fancyhdr}
\usepackage{fancybox}
\usepackage{empheq}
\usepackage{nameref}
\usepackage{multicol}
%\usepackage{emptypage}
%\usepackage{hyperref}
\usepackage[version=3]{mhchem}
%\usepackage{chmst-pdf}
\usepackage{xymtexpdf}
\usepackage{chemfig}
\usepackage{chemmacros}
\usepackage{listings}

\usepackage{tikz}
\usetikzlibrary{shapes.arrows,arrows,backgrounds,positioning,fit,%
                decorations.pathmorphing,decorations.pathreplacing,%
		decorations.text,calc,%
		shapes.geometric,shadows,%
		pgfplots.groupplots,%
	        intersections}

\usepackage{pgfplots}
% background color definition from pgfmanual-en-macros.tex
\definecolor{graphicbackground1}{rgb}{0.96,0.96,0.8}
% key to change color
\pgfkeys{/tikz/.cd,
	background color/.initial=graphicbackground1,
	background color/.get=\backcol,
	background color/.store in=\backcol,
}
\tikzset{background rectangle/.style={
		fill=\backcol,
	},
	use background/.style={
		show background rectangle
	}
}

\begin{document}

\title{Partícula En Un Pozo De Potencial Rectangular}
\author{José A. Navarro Ramón\\
Murcia, (España)\\
\texttt{josea.navarro@murciaeduca.es}
}

\maketitle

\begin{abstract}
Se analizan los estados posibles en los que se puede encontrar una partícula
en un pozo de potencial rectangular. También se desarrollan cálculos
numéricos y se obtienen gráficos.
\end{abstract}

\tableofcontents

% texto/ecuacion_schrodinger.tex

\chapter[Ecuación de Schrödinger para el pozo de potencial rectangular]
{Ecuación de Schrödinger unidimensional para el pozo de potencial rectangular}

\section{Pozo de potencial rectangular}
Supondremos que tenemos una partícula de masa $m$, que se encuentra en un
espacio de una dimensión $x$. La energía potencial en todo el espacio vale
$U(x)=U_0$, excepto en una región comprendida en el intervalo $0 < x < l$, donde
$l$ es una coordenada positiva; la energía potencial en esta región vale cero:
$U(x)=0$.

A continuación se presenta una representación del pozo de potencial.

\begin{figure}[ht]
\centering
\begin{tikzpicture}
[
scale=0.75,
%use background
]

\begin{axis}
[xmin=-10, xmax=16, ymin=-4, ymax=3,
xlabel={Posición}, ylabel={Energía},
xtick={0,6}, ytick={-3,0},
xticklabels={0,$l$}, yticklabels={0,$U_0$},
use background % Color definido en la cabecera del documento todo el gráfico
%background color=orange!20, use background, % Color todo el gráfico
%axis background/.style={fill=yellow!20} % Color en el interior de la gráfica
]
\addplot+[sharp plot,no marks,line width=2pt,color=black] coordinates
{(-10,0) (0,0) (0,-3) (6,-3) (6,0) (16,0)};
\end{axis}
\end{tikzpicture}
\caption{Representación del pozo de potencial rectangular}
\end{figure}

Reconocemos tres regiones:
\begin{itemize}
\item Zona I: Posiciones a la izquierda del pozo, $x<0$.
\item Zona II: Posiciones del pozo de potencial, $0<x<l$, donde la
partícula siente una atracción representada por el potencial.
\item Zona III: Posiciones a la derecha del pozo, $x>l$.
\end{itemize}

El pozo de potencial representa la región del eje $x$ donde la partícula
está sujeta a una fuerza atractiva.

\section{Variables adimensionales}
Se podría resolver la ecuación de Schrödinger directamente, utilizando las
variables en el SI, pero vamos a utilizar variables adimensionales porque,
en cierto sentido, los resultados que obtengamos son independientes de
las escalas de masas, longitudes y energías de cada caso particular.
Por supuesto, si las variables son adimensionales también lo será la
ecuación de Scrödinger.

Se utilizará una posición reducida adimensional, $x_r$, definida como:
\begin{equation}
\label{reduced_x}
	x_r = \frac{x}{C}
\end{equation}
$C$ es una constante con dimensiones de longitud, $[C] = L^1$.

De manera similar, necesitamos una energía reducida adimensional, $\epsilon_r$:
\begin{equation}
\label{reduced_e}
	\epsilon_r = \frac{E}{D}
\end{equation}
donde $D$ es una constante con dimensiones de energía, $[D] = ML^2T^{-2}$.

Estas constantes $C$ and $D$ deben depender de magnitudes que aparecen en la
ecuación de Schrödinger en una dimensión, $x$:
\begin{equation}
\label{schro1}
-\frac{\hbar^2}{2m}\frac{d\psi(x)}{dx^2}+U(x)\,\psi(x) = E\,\psi(x)
\end{equation}

Las magnitudes de las que dependerán las constantes anteriores son la
constate de Planck reducida, $\hbar$, la masa de la partícula, $m$, y el
valor del potencial que rodea al pozo, $U_0$.

Sus dimensiones respectivas son acción, masa y energía:
\begin{align*}
&[\hbar] = ML^2T^{-1}\\
&[m] = M\\
&[U_0] = ML^2T^{-2}
\end{align*}

La constante $C$ es una combinación de estas constantes
\[
[C] = \hbar^a \,m^b \,V_0^c =
(ML^2T^{-1})^a \,(M)^b \,(ML^2T^{-2})^c =
M^{a+b+c} \,L^{2a+2c} \,T^{-a-2c}
\]

Los exponentes $a$, $b$ and $c$ se pueden calcular sabiendo que $C$ sólo tiene
dimensiones de longitud ($M^0 L^1 T^0 = L$).
De esta manera se obtiene el siguiente sistema de ecuaciones:
\begin{align*}
&a + b + c = 0\\
&2a+2c = 1\\
&-a-2c = 0
\end{align*}

Las soluciones son: $a=1$ and $b=c=-1/2$. Así, $C$ vale:
\begin{equation}
C = \hbar \,(mU_0)^{-1/2} = \frac{\hbar}{\sqrt{mU_0}}
\end{equation}

El razonamiento del valor de la constante $D$ es más sencillo:
$D=U_0$, puesto que ambas magnitudes, $D$ y $U_0$ tienen dimensiones de
energía.

Así, tenemos:
\begin{equation}
\label{reduced_x2}
x_r = \frac{x}{C}=\frac{x}{\hbar/\sqrt{mU_0}} = \frac{\sqrt{mU_0}}{\hbar}\,x
\end{equation}

y
\begin{equation}
\label{reduced_e2}
\epsilon_r = \frac{E}{D}=\frac{E}{U_0}
\end{equation}

De esta manera, cuando queramos recuperar los valores reales (con las
dimensiones apropiadas), realizaremos los siguientes cálculos:
\[
x = \frac{\hbar}{\sqrt{mU_0}}\,x_r
\]

y
\[
E = U_0\,\epsilon_r
\]

\subsection{Función de onda y ecuación de Schrödinger reducida}
Nótese que $|\psi(x)|^2\,dx$ es una probabilidad, por lo que no tiene
dimensiones. De esto deducimos que la función de onda, $\psi(x)$, debe tener
dimensiones de $L^{-1/2}$.

Aprovechando que la constante $C$ tiene dimensiones de longitud, podemos
definir una función de onda reducida, $\psi_r(x_r)$:
\begin{equation}
\label{reduced_psi}
\psi_r(x_r) = \frac{\psi(x)}{C^{-1/2}}
\end{equation}

Para deducir la ecuación de Schrödinger reducida correspondiente, debemos
partir de la ecuación normal. Transformando ligeramente la \ref{schro1}:
\begin{equation}
\label{schro2}
\frac{\hbar^2}{2m}\frac{d^2\psi(x)}{dx^2} + (E-U(x))\psi(x) = 0
\end{equation}

Se calcula la primera derivada de la función de onda, $d\psi(x)/dx$, 
usando \ref{reduced_psi}:
\[
\frac{d\psi(x)}{dx}
= \frac{d}{dx}\, C^{-1/2}\psi_r(x_r)
= C^{-1/2}\,\frac{d\psi_r(x_r)}{dx}
= C^{-1/2}\,\frac{d\psi_r(x_r)}{dx_r}\,\frac{dx_r}{dx}
\]

De \ref{reduced_x}, podemos calcular $dx_r/dx$, lo que da $C^{-1}$:
\[
\frac{d\psi(x)}{dx}
= C^{-1/2}\,\frac{d\psi_r(x_r)}{dx_r}\,C^{-1}
= C^{-3/2}\,\frac{d\psi_r(x_r)}{dx_r}
\]

Ahora se calcula la segunda derivada, $d^2\psi(x)/dx^2$:
\begin{align*}
\frac{d^2\psi(x)}{dx^2}
&= \frac{d}{dx}\left[\frac{d\psi_r(x_r)}{dx}\right]
= \frac{d}{dx}\left[C^{-3/2}\,\frac{d\psi_r(x_r)}{dx_r}\right]
= C^{-3/2}\,\frac{d}{dx}\,\left[\frac{d\psi_r(x_r)}{dx_r}\right]\\
&= C^{-3/2}\,\frac{d}{dx_r}\,\left[\frac{d\psi_r(x_r)}{dx_r}\right]
	\,\frac{dx_r}{dx}
= C^{-3/2}\,\frac{d^2\psi_r(x_r)}{dx_r^2} \,C^{-1}\\
&= C^{-5/2}\,\frac{d^2\psi_r(x_r)}{dx_r^2}
\end{align*}

\subsection{Ecuación de Schrödinger para las zonas I y III}
En las zonas I y III la energía de potencial es cero.
Substituimos este valor del potencial en la ecuación de Schrödinger,
\ref{schro2},usando la energía reducida, \ref{reduced_e2}, y la función de
onda reducida, \ref{reduced_psi}:
\[
C^{-5/2}\,\frac{d^2\psi_r}{dx_r^2}
	 + \frac{2m}{\hbar^2}\,(\epsilon_r U_0-U_0)\,C^{-1/2}\psi_r=0
\]

Multiplicamos ambos términos por $C^{1/2}$:
\[
C^{-5/2}C^{1/2}\,\frac{d^2\psi_r}{dx_r^2}
	 - \frac{2mU_0}{\hbar^2}\,(1-\epsilon_r)\,C^{-1/2}\,C^{1/2}\,\psi_r=0
\]

\[
C^{-2}\,\frac{d^2\psi_r}{dx_r^2}
	 - \frac{2mU_0}{\hbar^2}\,(1-\epsilon_r)\,\psi_r=0
\]

De \ref{reduced_x2}, podemos deducir que $C^{-2}=mU_0/\hbar^2$:
\[
\frac{mU_0}{\hbar^2}\,\frac{d^2\psi_r}{dx_r^2}
	 - \frac{2mU_0}{\hbar^2}\,(1-\epsilon_r)\,\psi_r=0
\]

La ecuación de estado reducida para las zonas I y III es:
\begin{equation}
\frac{d^2\psi_r}{dx_r^2} + 2\,(1-\epsilon_r)\,\psi_r=0
\end{equation}

\subsection{Ecuación de Schrödinger para la zona II}
En la zona II la energía potencial es cero: $U(x) = 0$.
La ecuación de Schrödinger es:
\[
\label{schro2}
\frac{\hbar^2}{2m}\frac{d^2\psi(x)}{dx^2} + E\,\psi(x) = 0
\]

\[
C^{-5/2}\,\frac{d^2\psi_r}{dx_r^2}
	 + \frac{2m}{\hbar^2}\,\epsilon_r\,U_0\,C^{-1/2}\psi_r=0
\]

Multiplicamos ambos términos por $C^{1/2}$:
\[
C^{-5/2}C^{1/2}\,\frac{d^2\psi_r}{dx_r^2}
	 + \frac{2mU_0}{\hbar^2}\,\epsilon_r\,C^{-1/2}\,C^{1/2}\,\psi_r=0
\]

\[
C^{-2}\,\frac{d^2\psi_r}{dx_r^2}
	 + \frac{2mU_0}{\hbar^2}\,\epsilon_r\,\psi_r=0
\]

De \ref{reduced_x2}, podemos deducir que $C^{-2}=m|U_0|/\hbar^2$:
\[
\frac{mU_0}{\hbar^2}\,\frac{d^2\psi_r}{dx_r^2}
	 + \frac{2mU_0}{\hbar^2}\,\epsilon_r\,\psi_r=0
\]

As we did in the previous subsection, we get the reduced Schrödinger equation:
\begin{equation}
\frac{d^2\psi_r}{dx_r^2} + 2\epsilon_r\,\psi_r=0
\end{equation}


% texto/estados_ligados.tex

\section{Estados enlazados}
Los estados ligados o enlazados son aquellos en los que una partícula
no es libre, se encuentra en el interior del pozo de potencial y su energía
total es inferior a la energía potencial que rodea al pozo; en este caso
$E<U_0$; en términos de variables reducidas, $\epsilon_r < 1$.
En un estado enlazado la partícula no podría escapar del pozo según la Física
clásica.

\subsection{Primeros pasos en la resolución}
\subsubsection{Zona I}
\noindent
Dominio: $x_r < 0$.
Ecuación de Schrödinger reducida:
$\psi''_{rI} - 2(1-\epsilon_r)\,\psi_{rI} = 0$.
Ecuación característica usando $\psi_{rI}=e^{sx_r}$:
\[
s^2-2(1-\epsilon_r)=0
\]
\[
s^2=2(1-\epsilon_r)
\]
\[
s = \pm\sqrt{2(1-\epsilon_r)}
\]
So:
\[
\psi_{rI}(x_r)
 = C_1\,e^{-\sqrt{2\epsilon_r}\,x_r} + C_2\,e^{\sqrt{2\epsilon_r}\,x_r}
\]

Primera condición límite, la función de onda debe anularse en $-\infty$:
\[
\lim_{\scriptsize x_r\to\, -\infty} \psi_{rI}(x_r) = 0
\]

Deducimos que $C_1$ debe valer cero. La función de onda en la zona I es:
\[
\psi_{rI}(x_r) = C_2\,e^{\sqrt{2(1-\epsilon_r)}\,x_r}
\]

\subsubsection{Zona II}
\noindent
Dominio: $0 < x_r < l_r$.\\
Ecuación de Schrödinger reducida: $\psi''_{rII} - 2\epsilon_r\psi_{rII}= 0$.\\
Ecuación característica usando $\psi_{rII}=e^{sx_r}$:
\[
s^2+2\epsilon_r=0
\]
\[
s^2=-2\epsilon_r
\]
\[
s = \pm i\sqrt{2\epsilon_r}
\]

Así:
\begin{align*}
\psi_{rII}(x_r)
&=C_3\,e^{-i\,\sqrt{2\epsilon_r}\,x_r} + C_4\,e^{+i\,\sqrt{2\epsilon_r}\,x_r}\\
&=C_3\,\cos\left(\sqrt{2\epsilon_r}\,x_r\right)
 - i C_3\,\sen\left(\sqrt{2\epsilon_r}\,x_r\right)
 + C_4\,\cos\left(\sqrt{2\epsilon_r}\right)\,x_r 
 + i C_4\,\sen\left(\sqrt{2\epsilon_r}\,x_r\right)\\
&=(C_3+C_4)\,\cos\left(\sqrt{2\epsilon_r}\,x_r\right)
 + i (C_4-C_3)\,\sen\left(\sqrt{2\epsilon_r}\,x_r\right)
\end{align*}

La función de onda en la zona II es:
\[
\psi_{rII}(x_r)
= A\,\cos\left(\sqrt{2\epsilon_r}\,x_r\right)
 + B\,\sen\left(\sqrt{2\epsilon_r}\,x_r\right)
\]

\subsubsection{Zona III}
\noindent
Dominio: $x_r > l_r$.\\
Ecuación de Schrödinger reducida:
$\psi''_{rIII} - 2(1-\epsilon_r)\psi_{rIII}= 0$.\\
Ecuación característica usando $\psi_{rIII}=e^{sx_r}$:
\[
s^2-2(1-\epsilon_r)=0
\]
\[
s^2=2(1-\epsilon_r)
\]
\[
s = \pm\sqrt{2(1-\epsilon_r)}
\]
Así:
\[
\psi_{rIII}(x_r)
 = C_5\,e^{-\sqrt{2(1-\epsilon_r)}\,x_r} + C_6\,e^{\sqrt{2(1-\epsilon_r)}\,x_r}
\]

Teniendo en cuenta que la función de onda se debe anular en $+\infty$:
\[
\lim_{\scriptsize x_r\to\, +\infty} \psi_{rIII}(x_r) = 0
\]

Deducimos que $C_6$ debe valer cero. La función de onda en la zona III es:
\[
\psi_{rIII}(x_r) = C_5\,e^{\sqrt{2(1-\epsilon_r)}\,x_r}
\]

\subsection{Otras condiciones límite}

Las funciones de onda en las zonas I y II deben tender al mismo valor en
$x_r=0$.
\[
\lim_{\scriptsize x_r\to\,0} \psi_{rI}(x_r)
=
\lim_{\scriptsize x_r\to\,0} \psi_{rII}(x_r)
\]

Así:
\[
C_2 = A
\]

Las derivadas de las funciones de onda en I y en II también deben tender al
mismo valor en $x_r=0$:
\[
\lim_{\scriptsize x_r\to\,0} \psi'_{rI}(x_r)
=
\lim_{\scriptsize x_r\to\,0} \psi'_{rII}(x_r)
\]

Calculemos las derivadas de las funciones de onda en las zonas I y II:
\begin{align*}
\psi'_{rI} &= \frac{d\psi_{rI}(x_r)}{dx_r}
  = C_2\,\sqrt{2(1-\epsilon_r)}\,e^{\sqrt{2(1-\epsilon_r)}\,x_r}
  = A\,\sqrt{2(1-\epsilon_r)}\,e^{\sqrt{2(1-\epsilon_r)}\,x_r}\\
\psi'_{rII} &= \frac{d\psi_{rII}(x_r)}{dx_r}
  = -A\,\sqrt{2\epsilon_r}\,\sen\left(\sqrt{2\epsilon_r}\,x_r\right)
  + B\,\sqrt{2\epsilon_r}\,\cos\left(\sqrt{2\epsilon_r}\,x_r\right)
\end{align*}

Luego:
\[
A\,\sqrt{2(1-\epsilon_r)} =  B\,\sqrt{2\epsilon_r}
\]
\[
B = \sqrt{\frac{1-\epsilon_r}{\epsilon_r}}\,A
\]

Las funciones de onda en las zonas II y III deben tender al mismo valor en
en el otro extremo del pozo de potencial, $x_r=l_r$:
\[
\lim_{\scriptsize x_r\to\,l_r} \psi_{rII}(x_r)
=
\lim_{\scriptsize x_r\to\,l_r} \psi_{rIII}(x_r)
\]

\[
A\,\cos\left(l_r\sqrt{2\epsilon_r}\right)
 + A\,\sqrt{\frac{1-\epsilon_r}{\epsilon_r}}
   \,\sen\left(l_r\,\sqrt{2\epsilon_r}\right)
=
C_5\,\,e^{-l_r\sqrt{2(1-\epsilon_r}}
\]

Despejando $C_5$:
\begin{equation}
\label{c5_1}
C_5 = A\,e^{l_r\sqrt{2(1-\epsilon_r)}}
  \left\lbrace\cos\left(l_r\sqrt{2\epsilon_r}\right) 
  + \sqrt{\frac{1-\epsilon_r}{\epsilon_r}}\,
         \sen\left(l_r\sqrt{2\epsilon_r}\right)\right\rbrace
\end{equation}

\subsection{Ecuación de valores propios}
Las derivadas de la función de onda en las zonas II y III deben coincidir en
$x_r=l_r$:
\[
\lim_{\scriptsize x_r\to\,l_r} \psi'_{rII}(x_r)
=
\lim_{\scriptsize x_r\to\,l_r} \psi'_{rIII}(x_r)
\]

La derivada de $\psi_{rIII}$ es:
\[
\psi'_{rIII}(x_r) = \frac{d\psi_{rIII}(x_r)}{dx_r}
  = -C_5\,\sqrt{2(1-\epsilon_r)}\,e^{-\sqrt{2(1-\epsilon_r)}\,x_r}
\]

Entonces:
\[
-A\,\sqrt{2\epsilon_r}\,\sen\left(l_r\sqrt{2\epsilon_r}\right)
 + A\,\sqrt{\frac{1-\epsilon_r}{\epsilon_r}}\,\sqrt{2\epsilon_r}
   \,\cos\left(l_r\,\sqrt{2\epsilon_r}\right)
=
-C_5\,\sqrt{2(1-\epsilon_r)}\,e^{-l_r\sqrt{2(1-\epsilon_r)}}
\]

Despejando $C_5$:
\begin{equation}
\label{c5_2}
C_5 = A\,e^{l_r\sqrt{2(1-\epsilon_r)}}\,\sqrt{\frac{\epsilon_r}{1-\epsilon_r}}
  \left\lbrace\sen\left(l_r\sqrt{2\epsilon_r}\right)\,
  - \sqrt{\frac{1-\epsilon_r}{\epsilon_r}}\,
       \cos\left(l_r\sqrt{2\epsilon_r}\right)\right\rbrace
\end{equation}

Las expresiones \ref{c5_1} y \ref{c5_2} son iguales. Simplificando:
\[
\cos\left(l_r\sqrt{2\epsilon_r}\right) 
  + \sqrt{\frac{1-\epsilon_r}{\epsilon_r}}\,
         \sen\left(l_r\sqrt{2\epsilon_r}\right)
=
\sqrt{\frac{\epsilon_r}{1-\epsilon_r}}
  \left\lbrace\sen\left(l_r\sqrt{2\epsilon_r}\right)\,
  - \sqrt{\frac{1-\epsilon_r}{\epsilon_r}}\,
       \cos\left(l_r\sqrt{2\epsilon_r}\right)\right\rbrace
\]

Pasando a la izquierda la raíz cuadrada que está junto al signo igual:
\[
\sqrt{\frac{1-\epsilon_r}{\epsilon_r}}
\cos\left(l_r\sqrt{2\epsilon_r}\right)\,
  + \frac{1-\epsilon_r}{\epsilon_r}\,
         \sen\left(l_r\sqrt{2\epsilon_r}\right)
=
  \sen\left(l_r\sqrt{2\epsilon_r}\right)\,
  - \sqrt{\frac{1-\epsilon_r}{\epsilon_r}}\,
       \cos\left(l_r\sqrt{2\epsilon_r}\right)
\]

Pasando los cosenos a la izquierda y los senos a la derecha:
\[
2\sqrt{\frac{1-\epsilon_r}{\epsilon_r}}
\cos\left(l_r\sqrt{2\epsilon_r}\right) 
=
  \sen\left(l_r\sqrt{2\epsilon_r}\right)\,
  - \frac{1-\epsilon_r}{\epsilon_r}\,
       \sen\left(l_r\sqrt{2\epsilon_r}\right)
\]

\[
2\sqrt{\frac{1-\epsilon_r}{\epsilon_r}}
\cos\left(l_r\sqrt{2\epsilon_r}\right) 
=
  \left[1-\frac{1-\epsilon_r}{\epsilon_r}\right]
  \sen\left(l_r\sqrt{2\epsilon_r}\right)
\]

\[
2\sqrt{\frac{1-\epsilon_r}{\epsilon_r}}
\cos\left(l_r\sqrt{2\epsilon_r}\right) 
=
  \frac{2\epsilon_r-1}{\epsilon_r}
  \sen\left(l_r\sqrt{2\epsilon_r}\right)
\]

La ecuación de valores propios es:
\begin{equation}
\label{eigenvalue_eq}
\tan\left(l_r\sqrt{2\epsilon_r}\right)
=
\frac{2\epsilon_r}{2\epsilon_r-1}
\sqrt{\frac{1-\epsilon_r}{\epsilon_r}}
\end{equation}

Esta ecuación no se puede resolver exactamente, sino que debemos utilizar
métodos numéricos.

\subsection{Funciones propias}
Resumiendo y teniendo en cuenta los posibles valores propios, $\epsilon_r$,
la función de onda en cada zona es:
\begin{equation}
\label{wavefunctions}
\psi_r(x_r)=
\begin{cases}
A\,e^{\sqrt{2(1-\epsilon_r)}\,x_r}
& -\infty<x_r<0\\
A\,\left[\cos\left(\sqrt{2\epsilon_r}\,x_r\right)
  + \sqrt{\frac{1-\epsilon_r}{\epsilon}}\,
    \sen\left(\sqrt{2\epsilon_r}\,x_r\right)\right]
& 0<x_r<l_r\phantom{\rule[-9pt]{0em}{0em}}\\
A\,\left[\cos\left(l_r\sqrt{2\epsilon_r}\right)
  + \sqrt{\frac{1-\epsilon_r}{\epsilon}}\,
    \sen\left(l_r\sqrt{2\epsilon_r}\right)\right]
    e^{-\sqrt{2(1-\epsilon_r)}\,(x_r-l_r)}
& l_r<x_r<+\infty
\end{cases}
\end{equation}

\subsection{Cuadrados de la función de onda en cada zona}
Necesitamos calcular el cuadrado de la función de onda pues tenemos
que normalizarla:

\subsubsection{Zone I}
\[
\left|\psi_{rI}(x_r)\right|^2
= A^2\left(e^{\sqrt{2(1-\epsilon_r)}\,x_r}\right)^2
= A^2\,e^{2\sqrt{2(1-\epsilon_r)}\,x_r}
\]

\subsubsection{Zone II}
\begin{align*}
\left|\psi_{rII}(x_r)\right|^2
&= A^2\,\left\lbrace\cos\left(\sqrt{2\epsilon_r}\,x_r\right)
  + \sqrt{\frac{1-\epsilon_r}{\epsilon_r}}\,
    \sen\left(\sqrt{2\epsilon_r}\,x_r\right)
\right\rbrace^2\\
&= A^2\,\left\lbrace
 \cos^2\left(\sqrt{2\epsilon_r}\,x_r\right)\,
 + \frac{1-\epsilon_r}{\epsilon_r}\,\sen^2\left(\sqrt{2\epsilon_r}\,x_r\right)
 + 2\sqrt{\frac{1-\epsilon_r}{\epsilon_r}}\,
 \sen\left(\sqrt{2\epsilon_r}\,x_r\right)\,
 \cos\left(\sqrt{2\epsilon_r}\,x_r\right)
\right\rbrace\\
&= A^2\,\left\lbrace
 1 - \sen^2\left(\sqrt{2\epsilon_r}\,x_r\right)\,
 + \frac{1-\epsilon_r}{\epsilon_r}\,\sen^2\left(\sqrt{2\epsilon_r}\,x_r\right)
 + \sqrt{\frac{1-\epsilon_r}{\epsilon_r}}\,
 \sen\left(2\sqrt{2\epsilon_r}\,x_r\right)
\right\rbrace\\
&= A^2\,\left\lbrace
 1  
 + \left(-1+\frac{1-\epsilon_r}{\epsilon_r}\right)\,\sen^2\left(\sqrt{2\epsilon_r}\,x_r\right)
 + \sqrt{\frac{1-\epsilon_r}{\epsilon_r}}\,
 \sen\left(2\sqrt{2\epsilon_r}\,x_r\right)
\right\rbrace\\
&= A^2\,\left\lbrace
 1  
 + \frac{1-2\epsilon_r}{\epsilon_r}\,\sen^2\left(\sqrt{2\epsilon_r}\,x_r\right)
 + \sqrt{\frac{1-\epsilon_r}{\epsilon_r}}\,
 \sen\left(2\sqrt{2\epsilon_r}\,x_r\right)
\right\rbrace
\end{align*}

\subsubsection{Zone III}

\begin{align*}
\left|\psi_{rIII}(x_r)\right|^2
&= A^2\,\left\lbrace\cos\left(l_r\sqrt{2\epsilon_r}\right)
  + \sqrt{\frac{1-\epsilon_r}{\epsilon_r}}\,
    \sen\left(l_r\sqrt{2\epsilon_r}\right)
\right\rbrace^2
\left\lbrace
\,e^{-\sqrt{2(1-\epsilon_r)}\,(x_r-l_r)}
\right\rbrace^2\\
&= A^2\,\left\lbrace
 \cos^2\left(l_r\sqrt{2\epsilon_r}\right)\,
 + \frac{1-\epsilon_r}{\epsilon_r}\,\sen^2\left(l_r\sqrt{2\epsilon_r}\right)
 + \sqrt{\frac{1-\epsilon_r}{\epsilon_r}}\,
 \sen\left(2l_r\sqrt{2\epsilon_r}\right)
\right\rbrace
\,e^{-2\sqrt{2(1-\epsilon_r)}\,(x_r-l_r)}\\
&= A^2\,\left\lbrace
 1 - \sen^2\left(l_r\sqrt{2\epsilon_r}\right)\,
 + \frac{1-\epsilon_r}{\epsilon_r}\,\sen^2\left(l_r\sqrt{2\epsilon_r}\right)
 + \sqrt{\frac{1-\epsilon_r}{\epsilon_r}}\,
 \sen\left(2l_r\sqrt{2\epsilon_r}\right)
\right\rbrace
\,e^{-2\sqrt{2(1-\epsilon_r)}\,(x_r-l_r)}\\
&= A^2\,\left\lbrace
 1 
 + \left(-1+\frac{1-\epsilon_r}{\epsilon_r}\right)\,\sen^2\left(l_r\sqrt{2\epsilon_r}\right)
 + \sqrt{\frac{1-\epsilon_r}{\epsilon_r}}\,
 \sen\left(2l_r\sqrt{2\epsilon_r}\right)
\right\rbrace
\,e^{-2\sqrt{2(1-\epsilon_r)}\,(x_r-l_r)}\\
&= A^2\,\left\lbrace
 1 
 + \frac{1-2\epsilon_r}{\epsilon_r}\,\sen^2\left(l_r\sqrt{2\epsilon_r}\right)
 + \sqrt{\frac{1-\epsilon_r}{\epsilon_r}}\,
 \sen\left(2l_r\sqrt{2\epsilon_r}\right)
\right\rbrace
\,e^{-2\sqrt{2(1-\epsilon_r)}\,(x_r-l_r)}\\
&= A^2\,e^{-2\sqrt{2(1-\epsilon_r)}\,(x_r-l_r)}
\end{align*}

Con respecto al último paso, nótese que:
\[
 \frac{1-2\epsilon_r}{\epsilon_r}\,\sen^2\left(l_r\sqrt{2\epsilon_r}\right)
 + \sqrt{\frac{1-\epsilon_r}{\epsilon_r}}\,
 \sen\left(2l_r\sqrt{2\epsilon_r}\right) = 0
\]
sólo si $\epsilon_r$ es un valor propio.  

Para demostrar esto, supondremos que $\epsilon_r$ es un valor propio, entonces
debe cumplir la ecuación de valores propios, \ref{eigenvalue_eq}:
\[
\tan\left(l_r\sqrt{2\epsilon_r}\right)
=
\frac{2\epsilon_r}{2\epsilon_r-1}
\sqrt{\frac{1-\epsilon_r}{\epsilon_r}}
\]

\[
\frac{\sen\left(l_r\sqrt{2\epsilon_r}\right)}
{\cos\left(l_r\sqrt{2\epsilon_r}\right)}
=
\frac{2\epsilon_r}{2\epsilon_r-1}
\sqrt{\frac{1-\epsilon_r}{\epsilon_r}}
\]

Reagrupando términos:
\[
\frac{2\epsilon_r-1}{\epsilon_r}\,
\sen\left(l_r\sqrt{2\epsilon_r}\right)
=
2\sqrt{\frac{1-\epsilon_r}{\epsilon_r}}
\cos\left(l_r\sqrt{2\epsilon_r}\right)
\]

\[
\frac{1-2\epsilon_r}{\epsilon_r}\,
\sen\left(l_r\sqrt{2\epsilon_r}\right)
+
2\sqrt{\frac{1-\epsilon_r}{\epsilon_r}}
\cos\left(l_r\sqrt{2\epsilon_r}\right)
=0
\]

Multiplicando ambos miembros por el seno del ángulo:
\[
\frac{1-2\epsilon_r}{\epsilon_r}\,
\sen^2\left(l_r\sqrt{2\epsilon_r}\right)
+
2\sqrt{\frac{1-\epsilon_r}{\epsilon_r}}
\sen\left(l_r\sqrt{2\epsilon_r}\right)\,
\cos\left(l_r\sqrt{2\epsilon_r}\right)
=0
\]

Entonces hemos demostrado que:
\[
\frac{1-2\epsilon_r}{\epsilon_r}\,
\sen^2\left(l_r\sqrt{2\epsilon_r}\right)
+
\sqrt{\frac{1-\epsilon_r}{\epsilon_r}}
\sen\left(2l_r\sqrt{2\epsilon_r}\right)
=0
\]

\subsection{Normalización}
La función de onda se tiene que normalizar. Esto es:
\[
\int_{-\infty}^{+\infty} \left|\psi_r(x_r)\right|^2 dx_r = 1
\]

\[
\int_{-\infty}^{0}\left|\psi_{rI}(x_r)\right|^2 dx_r 
 + \int_{0}^{l_r}\left|\psi_{rII}(x_r)\right|^2 dx_r 
 + \int_{l_r}^{+\infty}\left|\psi_{rIII}(x_r)\right|^2 dx_r 
 = 1
\]

Debemos calcular tres integrales:
\begin{equation}
\label{normalization_eq}
I_1 + I_2 + I_3 = 1
\end{equation}

\subsubsection{Primera integral: $I_1$}

\[
I_1 = \int_{-\infty}^{0}\left|\psi_{rI}(x_r)\right|^2 dx_r 
= 
A^2\,\int_{-\infty}^0 e^{2\sqrt{2(1-\epsilon_r)}\,x_r} dx_r
\]

Cambio de variable:
\[
2\sqrt{2(1-\epsilon_r)}\,x_r = t_r
\hspace{2em}
dx_r = \frac{1}{2\sqrt{2(1-\epsilon_r)}}\,dt_r
\]

\[
I_1
= A^2\,\frac{1}{2\sqrt{2(1-\epsilon_r)}}\int_{-\infty}^0 e^{t_r} dt_r
= A^2\,\frac{1}{2\sqrt{2(1-\epsilon_r)}}(e^0-e^{-\infty})
\]

Luego:
\begin{equation}
\label{I1}
I_1 = \frac{A^2}{2\sqrt{2(1-\epsilon_r)}}
\end{equation}

\subsubsection{Tercera integral: $I_3$}

Nos saltamos la segunda integral por el momento:
\[
I_3 = \int_{l_r}^{+\infty}\left|\psi_{rIII}(x_r)\right|^2 dx_r 
= 
A^2\,\int_{l_r}^{+\infty} e^{-2\sqrt{2(1-\epsilon_r)}\,(x_r-l_r)} dx_r
\]

Cambio de variable:
\[
-2\sqrt{2(1-\epsilon_r)}\,(x_r-l_r) = t_r
\hspace{2em}
dx_r = \frac{-1}{2\sqrt{2(1-\epsilon_r)}}\,dt_r
\]

\[
I_3
= \frac{-A^2}{2\sqrt{2(1-\epsilon_r)}}\int_{0}^{-\infty} e^{t_r} dt_r
= \frac{-A^2}{2\sqrt{2(1-\epsilon_r)}}(e^{-\infty}-e^{0})
\]

Entonces:
\begin{equation}
\label{I3}
I_3 = \frac{A^2}{2\sqrt{2(1-\epsilon_r)}}
\end{equation}

\subsubsection{Segunda integral: $I_2$}

Como veremos, esta integral se resolverá en tres partes,
$I_{21}$, $I_{22}$, $I_{23}$:
\[
I_2 = \int_{0}^{l_r}\left|\psi_{rII}(x_r)\right|^2 dx_r 
= 
A^2\,\int_{0}^{l_r}
\left\lbrace
1
 + \frac{1-2\epsilon_r}{\epsilon_r}\,\sen^2\left(\sqrt{2\epsilon_r}\,x_r\right)
 + \sqrt{\frac{1-\epsilon_r}{\epsilon_r}}\,\sen\left(2\sqrt{2\epsilon_r}\,x_r\right)
\right\rbrace
dx_r
\]

\[
I_2 = A^2\,\left\lbrace I_{21} + I_{22} + I_{23}\right\rbrace
\]

\subsubsection{Segunda integral, primera parte: $I_{21}$}

\[
I_{21}
= 
\int_{0}^{l_r} dx_r
\]

\begin{equation}
\label{I21}
I_{21} = l_r
\end{equation}

\subsubsection{Segunda integral, segunda parte: $I_{22}$}

\[
I_{22}
= 
\frac{1-2\epsilon_r}{\epsilon_r}\,
\int_{0}^{l_r}\sen^2\left(\sqrt{2\epsilon_r}\,x_r\right) dx_r
\]

Cambio de variable:
\[
\sqrt{2\epsilon_r}\,x_r = t_r
\hspace{2em}
dx_r = \frac{1}{\sqrt{2\epsilon_r}}\,dt_r
\]

\[
I_{22}
= 
\frac{1-2\epsilon_r}{\epsilon_r}\,
\frac{1}{\sqrt{2\epsilon_r}}\,
\int_{0}^{l_r\sqrt{2\epsilon_r}}\hspace*{-2em}\sen^2 t_r\, dt_r
\]

Escribamos $\sen^2 t_r$ de una forma diferente:
\[
\cos 2t_r = \cos^2 t_r - \sen^2 t_r = 1-\sen^2 t_r - \sen^2 t_r
= 1-2\sen^2 t_r
\]

\[
\sen^2 t_r = \frac{1}{2} - \frac{1}{2}\,\cos (2t_r)
\]

\[
I_{22}
= 
\frac{1-2\epsilon_r}{\epsilon_r\sqrt{2\epsilon_r}}\,
\left\lbrace
\frac{1}{2}\int_{0}^{l_r\sqrt{2\epsilon_r}}\hspace*{-2em}\,dt_r
- \frac{1}{2}\int_{0}^{l_r\sqrt{2\epsilon_r}}\hspace*{-2em}\cos(2t_r)\, dt_r
\right\rbrace
\]

Cambio de variable:
\[
2t_r = z_r
\hspace{2em}
dt_r = \frac{1}{2}\,dz_r
\]

\begin{align*}
I_{22}
&= 
\frac{1-2\epsilon_r}{\epsilon_r\sqrt{2\epsilon_r}}\,
\left\lbrace
\frac{l_r\sqrt{2\epsilon_r}}{2}
- \frac{1}{4}\int_{0}^{2l_r\sqrt{2\epsilon_r}}\hspace*{-2em}\cos(2t_r)\, dt_r
\right\rbrace\\
&= 
\frac{1-2\epsilon_r}{\epsilon_r\sqrt{2\epsilon_r}}\,
\left\lbrace
\frac{2l_r\sqrt{2\epsilon_r}}{4}
- \frac{\sen\left(2l_r\sqrt{2\epsilon_r}\right)}{4}
\right\rbrace
\end{align*}

\begin{equation}
\label{I22}
I_{22} =
\frac{1-2\epsilon_r}{4\epsilon_r\sqrt{2\epsilon_r}}\,
\left\lbrace
2l_r\sqrt{2\epsilon_r}
- \sen\left(2l_r\sqrt{2\epsilon_r}\right)
\right\rbrace
\end{equation}

\subsubsection{Segunda integral, tercera parte: $I_{23}$}

\[
I_{23}
= 
\sqrt{\frac{1-\epsilon_r}{\epsilon_r}}\,
\int_{0}^{l_r}\sen\left(2\sqrt{2\epsilon_r}\,x_r\right)\, dx_r
\]

Cambio de variable:
\[
2\sqrt{2\epsilon_r}\,x_r = t_r
\hspace{2em}
dx_r = \frac{1}{2\sqrt{2\epsilon_r}}\,dt_r
\]

\begin{align*}
I_{23}
&= 
\sqrt{\frac{1-\epsilon_r}{\epsilon_r}}\,
\frac{1}{2\sqrt{2\epsilon_r}}\,
\int_{0}^{2l_r\sqrt{2\epsilon_r}}\hspace*{-2em}\sen t_r\, dt_r\\
&= 
-\sqrt{\frac{1-\epsilon_r}{\epsilon_r}}\,
\frac{1}{2\sqrt{2\epsilon_r}}\,
\left[\cos\left(2l_r\sqrt{2\epsilon_r}\right) - cos 0\right]
\end{align*}

\begin{equation}
\label{I23}
I_{23} =
\sqrt{\frac{1-\epsilon_r}{\epsilon_r}}\,
\frac{1}{2\sqrt{2\epsilon_r}}\,
\left[1 - \cos\left(2l_r\sqrt{2\epsilon_r}\right)\right]
\end{equation}

So $I_2$ is:
\begin{equation}
\label{I2}
I_2 
= A^2\,
\left\lbrace
l_r
+
\frac{1-2\epsilon_r}{4\epsilon_r\sqrt{2\epsilon_r}}\,
\left[
2l_r\sqrt{2\epsilon_r}
- \sen\left(2l_r\sqrt{2\epsilon_r}\right)
\right]
+
\sqrt{\frac{1-\epsilon_r}{\epsilon_r}}\,
\frac{1}{2\sqrt{2\epsilon_r}}\,
\left[1 - \cos\left(2l_r\sqrt{2\epsilon_r}\right)\right]
\right\rbrace
\end{equation}

De la condición de normalización, \ref{normalization_eq}, podemos obtener el
valor de $A$:
\[
A = \frac{1}{\sqrt{I_1 + I_2 + I_3}}
\]


% texto/estados_no_ligados.tex

\chapter{Estados no ligados}
Los estados no ligados o enlazados son aquellos en los que una partícula
es libre, su energía total es mayor o igual a la energía potencial que rodea
al pozo; en este caso $E\ge 0$; en términos de variables reducidas, $\epsilon_r \ge 1$.
En un estado enlazado la partícula es libre para moverse por todo el espacio.

\subsection{Primeros pasos en la resolución}
\subsubsection{Zona I}
\noindent
Dominio: $x_r < 0$.
Ecuación de Schrödinger reducida:
$\psi''_{rI} - 2(1-\epsilon_r)\,\psi_{rI} = 0$.
Ecuación característica usando $\psi_{rI}=e^{sx_r}$:
\[
s^2-2(1-\epsilon_r)=0
\]
\[
s^2=2(1-\epsilon_r)
\]
\[
s = \pm\sqrt{2(1-\epsilon_r)}
\]

Así:
\[
\psi_{rI}(x_r)
 = C_1\,e^{-\sqrt{2\epsilon_r}\,x_r} + C_2\,e^{\sqrt{2\epsilon_r}\,x_r}
\]

La partícula puede moverse libremente por todo el espacio, por lo que la
función de onda no tiene que anularse cuando se aleja.

\subsubsection{Zona II}
\noindent
Dominio: $0 < x_r < l_r$.\\
Ecuación de Schrödinger reducida: $\psi''_{rII} - 2\epsilon_r\psi_{rII}= 0$.\\
Ecuación característica usando $\psi_{rII}=e^{sx_r}$:
\[
s^2+2\epsilon_r=0
\]
\[
s^2=-2\epsilon_r
\]
\[
s = \pm i\sqrt{2\epsilon_r}
\]

Así:
\begin{align*}
\psi_{rII}(x_r)
&=C_3\,e^{-i\,\sqrt{2\epsilon_r}\,x_r} + C_4\,e^{+i\,\sqrt{2\epsilon_r}\,x_r}\\
&=C_3\,\cos\left(\sqrt{2\epsilon_r}\,x_r\right)
 - i C_3\,\sen\left(\sqrt{2\epsilon_r}\,x_r\right)
 + C_4\,\cos\left(\sqrt{2\epsilon_r}\right)\,x_r 
 + i C_4\,\sen\left(\sqrt{2\epsilon_r}\,x_r\right)\\
&=(C_3+C_4)\,\cos\left(\sqrt{2\epsilon_r}\,x_r\right)
 + i (C_4-C_3)\,\sen\left(\sqrt{2\epsilon_r}\,x_r\right)
\end{align*}

La función de onda en la zona II es:
\[
\psi_{rII}(x_r)
= A\,\cos\left(\sqrt{2\epsilon_r}\,x_r\right)
 + B\,\sen\left(\sqrt{2\epsilon_r}\,x_r\right)
\]

\subsubsection{Zona III}
\noindent
Dominio: $x_r > l_r$.\\
Ecuación de Schrödinger reducida:
$\psi''_{rIII} - 2(1-\epsilon_r)\psi_{rIII}= 0$.\\
Ecuación característica usando $\psi_{rIII}=e^{sx_r}$:
\[
s^2-2(1-\epsilon_r)=0
\]
\[
s^2=2(1-\epsilon_r)
\]
\[
s = \pm\sqrt{2(1-\epsilon_r)}
\]

Así:
\[
\psi_{rIII}(x_r)
 = C_5\,e^{-\sqrt{2(1-\epsilon_r)}\,x_r} + C_6\,e^{\sqrt{2(1-\epsilon_r)}\,x_r}
\]

\subsection{Condiciones límite}

Las funciones de onda en las zonas I y II deben tender al mismo valor en
$x_r=0$.
\[
\lim_{x_r\to\,0} \psi_{rI}(x_r)
=
\lim_{x_r\to\,0} \psi_{rII}(x_r)
\]

Así:
\[
C_2 = A
\]

Las derivadas de las funciones de onda en I y en II también deben tender al
mismo valor en $x_r=0$:
\[
\lim_{x_r\to\,0} \psi'_{rI}(x_r)
=
\lim_{x_r\to\,0} \psi'_{rII}(x_r)
\]

Calculemos las derivadas de las funciones de onda en las zonas I y II:
\begin{align*}
\psi'_{rI} &= \frac{d\psi_{rI}(x_r)}{dx_r}
  = C_2\,\sqrt{2(1-\epsilon_r)}\,e^{\sqrt{2(1-\epsilon_r)}\,x_r}
  = A\,\sqrt{2(1-\epsilon_r)}\,e^{\sqrt{2(1-\epsilon_r)}\,x_r}\\
\psi'_{rII} &= \frac{d\psi_{rII}(x_r)}{dx_r}
  = -A\,\sqrt{2\epsilon_r}\,\sen\left(\sqrt{2\epsilon_r}\,x_r\right)
  + B\,\sqrt{2\epsilon_r}\,\cos\left(\sqrt{2\epsilon_r}\,x_r\right)
\end{align*}

Luego:
\[
A\,\sqrt{2(1-\epsilon_r)} =  B\,\sqrt{2\epsilon_r}
\]
\[
B = \sqrt{\frac{1-\epsilon_r}{\epsilon_r}}\,A
\]

Las funciones de onda en las zonas II y III deben tender al mismo valor en
en el otro extremo del pozo de potencial, $x_r=l_r$:
\[
\lim_{x_r\to\,l_r} \psi_{rII}(x_r)
=
\lim_{x_r\to\,l_r} \psi_{rIII}(x_r)
\]

\[
A\,\cos\left(l_r\sqrt{2\epsilon_r}\right)
 + A\,\sqrt{\frac{1-\epsilon_r}{\epsilon_r}}
   \,\sen\left(l_r\,\sqrt{2\epsilon_r}\right)
=
C_5\,\,e^{-l_r\sqrt{2(1-\epsilon_r}}
\]

Despejando $C_5$:
\begin{equation}
\label{c5_1}
C_5 = A\,e^{l_r\sqrt{2(1-\epsilon_r)}}
  \left\lbrace\cos\left(l_r\sqrt{2\epsilon_r}\right) 
  + \sqrt{\frac{1-\epsilon_r}{\epsilon_r}}\,
         \sen\left(l_r\sqrt{2\epsilon_r}\right)\right\rbrace
\end{equation}



% texto/solucion_numerica.tex

\chapter{Solución numérica de los estados ligados}

\section{Ecuación de valores propios}

La ecuación de valores propios, \ref{eigenvalue_eq}, tiene dos partes:

\[
\tg\left(l_r\sqrt{2\epsilon_r}\right)
=
\frac{2\epsilon_r}{2\epsilon_r-1}
\sqrt{\frac{1-\epsilon_r}{\epsilon_r}}
\]

\[
f_1(\epsilon_r)
=
f_2(\epsilon_r)
\]

Una es la función tangente:
\begin{equation}
\label{eigenval_func1}
f_1(\epsilon_r) = \tg\left({l_r\sqrt{2\epsilon_r}}\right)
\end{equation}

y la otra es una función no trigonométrica:
\begin{equation}
\label{eigenval_func2}
f_2(\epsilon_r) = \frac{2\epsilon_r}{2\epsilon_r-1}
\sqrt{\frac{1-\epsilon_r}{\epsilon_r}}
\end{equation}

\subsection{Función tangente}

\subsubsection{Función: $\tg\theta$}
La función tangente, $y = \tg\theta$ no es continua; tiene infinitas
asíntotas verticales:
\[
y = \tg\theta = \frac{\sen\theta}{\cos\theta}
\]

Estas asíntotas corresponden a los puntos del eje $x$ en los que $\cos\theta=0$.
Estos son los $\theta=(2n-1)\pi/2$, donde $n=\pm 1, \pm 2, \cdots$.
La función tangente por esta razón, discontinua; pero entre las asíntotas es
continua.

Ahora veamos los ceros de la función tangente. La tangente será cero cuando
el seno del ángulo sea cero. Esto ocurre en los puntos:
\begin{equation}
\label{asymptotes_tan}
\theta=\cdots,-3\pi,-2\pi,-\pi,0,\pi,2\pi,3\pi,\cdots
\end{equation}

Analicemos si es creciente o decreciente entre las asíntotas. Si la primera
derivada es positiva la función es creciente y decreciente si es negativa.
La primera derivada es creciente pues el cuadrado de un número real siempre
es positivo:
\[
y' = \frac{d(\tg\theta)}{d\theta} = \frac{1}{\cos^{2}\theta}
\]

La segunda derivada es:
\[
y'' = \frac{d(1/\cos^{2}\theta)}{d\theta} = 
\frac{-\sen(2\theta)}{\cos^{4}\theta}
\]

La tangente será cóncava si la segunda derivada es positiva y convexa
si esta segunda derivada fuera negativa.

Razonemos el signo. Por un lado, $cos^{4}\theta$ es siempre positivo
(potencia par).
La convexidad o concavidad dependerá del signo de $-\sen (2\theta)$.
Veamos:

Como la función tangente es periódica, vamos a razonar la concavidad y la
convexidad entre dos asíntotas consecutivas; lo que se deduzca se puede
generalizar al resto de intervalos.

La función $\sen\theta$ es negativa en el intervalo $(-\pi,0)$ y positiva en
$(0,\pi)$.

Se extender el razonamiento a la función $\sen (2\theta)$ teniendo en cuenta
que el dominio se divide entre dos, lo que significa que los valores de
$\theta$ anteriores se deben dividir entre dos.
Resumiendo, $\sen (2\theta)$ será negativa en el intervalo abierto $(-\pi/2,0)$
y positiva en $(0,\pi/2)$.

A $-\sen (2\theta)$ le ocurrirá lo contrario debido al signo negativo:
será positiva en el intervalo $(-\pi/2,0)$ y negativa en $(0,\pi/2)$.

Como $(-\pi/2,+\pi/2)$ forman un intervalo entre dos asíntotas consecutivas,
la primera mitad de éste es cóncavo y la segunda mitad convexo.

\begin{figure}[ht]
\centering
\begin{tikzpicture}
\begin{axis}[
restrict y to domain=-10:10,
samples=2000,
%width=7cm, height=5cm,
xmin=-3.4*pi/2-0.1, xmax=3.3*pi/2+0.1,
xtick={-4.7124,-1.5708,...,10},
xticklabels={$-\frac32 \pi$,$-\pi/2$,$\pi/2$,$\frac32 \pi$},
ytick={0},
yticklabels={0},
axis x line=center,
axis y line=center,
minor tick num=1,
use background
]

% Asíntotas
\foreach \i in {-3*pi/2,-pi/2,pi/2,3*pi/2}
{\addplot[gray!50,dashed] coordinates{(\i,-10) (\i,10)};};%

% Función tangente
\addplot[blue] gnuplot[id=tangens,domain=-3.4*pi/2:3.4*pi] {tan(x)};

\legend{$y=\tg(\theta)$}
\end{axis}
\end{tikzpicture}
\caption{La función tangente, $y=\tg{\theta}$ es periódica}
\end{figure}

\subsubsection{Función: $\tg\left(l_r\sqrt{2\epsilon_r}\right)$}
Muchas características generales estudiadas para $\tg\theta$ siguen siendo
válidas ahora, pero hay unos cambios:

El dominio ya no es el de todos los reales menos las posiciones donde hay
asíntotas; ahora la energía reducida debe estar comprendida entre cero y uno,
$0<\epsilon_r<1$, lo que limita el número de asíntotas que habrá.
Recordando los ángulos en los que la función tangente no está definida,
\ref{asymptotes_tan}, y descartando los valores negativos:

\[
\theta=l_r\sqrt{2\epsilon_r}=0,\pi,2\pi,3\pi,\cdots
\]

\[
l_r\sqrt{2\epsilon_r}=(2n-1)\,\frac{\pi}{2}
;\hspace*{1em}
n=1,2,3,\cdots
\]

El entero $n$ tiene un límite superior, el correspondiente al valor
$\epsilon_r=1$, que es inalcanzable:
\begin{equation}
\label{n_limsup}
n < \frac{\sqrt{2}}{\pi}\,l_r + \frac{1}{2}
\end{equation}

Por otra parte, este valor de n máximo posible nos da el número de valores
propios o de estados ligados.

Además no puede valer cero; su menor valor es uno: $n\ge 1$. Están
prohibidos los valores $n<1$. En este caso, para un infinitésimo $\delta$:
\[
1\le n\le\left\lfloor \frac{\sqrt{2}}{\pi}\,l_r+\frac{1}{2}-\delta\right\rfloor
\]

De aquí que:
\[
1 \le \left\lfloor\frac{\sqrt{2}}{\pi}\,l_r+\frac{1}{2}-\delta\right\rfloor
\]

\[
1 \le \frac{\sqrt{2}}{\pi}\,l_r+\frac{1}{2}-\delta
\]

\[
1 < \frac{\sqrt{2}}{\pi}\,l_r+\frac{1}{2}
\]

\[
\frac{1}{2} < \frac{\sqrt{2}}{\pi}\,l_r
\]

Así, para que haya estados ligados, $l_r$ debe ser superior a:
\[
l_r > \frac{\pi}{2\sqrt{2}} \approx 1,11072073\cdots
\]

Como la variable independiente de la función, $l_r\sqrt{2\epsilon_r}$,
es una función creciente de $\epsilon_r$, las asíntotas ya no están
equiespaciadas, sino que esta distancia aumenta (hasta que se termina el
dominio de la función con $\epsilon_r=1$). Por ejemplo, para un
valor $l_r=11,45575$ se puede ver una representación en la figura
~\ref{fig:eigenvalfuncs12}.

\subsection{Función no trigonométrica}
La función no trigonométrica, \ref{eigenval_func2}:
\[
\frac{2\epsilon_r}{2\epsilon_r-1}
\sqrt{\frac{1-\epsilon_r}{\epsilon_r}}
\]
está representada en la figura ~\ref{fig:eigenvalfuncs12}.

Su dominio consiste en todos los reales comprendidos entre 0 y 1, excepto
$\epsilon_r=0.5$, que es la asíntota vertical de esta función.  

\begin{figure}[ht]
\centering
\begin{tikzpicture}
\begin{groupplot}[
group style={
  group name=mygroup,
  group size=2 by 1,
  xlabels at=edge bottom,
  ylabels at=edge left
},
footnotesize
]

\nextgroupplot[
restrict y to domain=-10:10,
samples=3000,
xmin=0,xmax=1,
ymin=-10,ymax=10,
width=7cm,
xlabel=$\epsilon_r$,
ytick={0},
yticklabels={0},
minor tick num=1,
use background
]
% Función tangente
\addplot[blue] gnuplot[id=tangenvar,domain=0:1] {tan(11.45575*sqrt(2*x))};

% Asíntotas
\foreach \i in {pi/2,pi/2,3*pi/2,5*pi/2,7*pi/2,9*pi/2}
{\addplot[gray!50,dashed] coordinates{(0.5*(\i/11.45575)^2,-10)(0.5*(\i/11.45575)^2,10)};};%
\legend{$f_1(\epsilon_r) =\tg(l_r\sqrt{2\epsilon_r)}$}

\nextgroupplot[
restrict y to domain=-50:50,
samples=2000,
width=7cm,
%width=7cm, height=5cm,
xmin=0, xmax=1.0,
ymin=-30,ymax=30,
ytick={0},
yticklabels={0},
%axis x line=center,
%axis y line=center,
xlabel=$\epsilon_r$,
minor tick num=1,
use background
]

% Función tangente
\addplot[red] gnuplot[id=tangenvar,domain=0:1] {2*x*sqrt((1-x)/x)/(2*x-1)};

\legend{
$f_2(\epsilon_r) = \frac{2\epsilon_r}{2\epsilon_r-1}\sqrt{\frac{1-\epsilon_r}{\epsilon_r}}$
}

% Asíntotas
\addplot[gray!50,dashed] coordinates{(0,-50) (0,50)};
\addplot[gray!50,dashed] coordinates{(0.5,-50) (0.5,50)};

\end{groupplot}
\end{tikzpicture}

\caption{Las dos funciones que forman la ecuación de valores propios}
\label{fig:eigenvalfuncs12}
\end{figure}

Los valores propios se encuentran en los puntos de corte de estas dos
funciones, ver figura ~\ref{fig:eigenvalues}:

\begin{figure}[ht]
\centering
\begin{tikzpicture}
[scale=0.5]
\begin{axis}[
restrict y to domain=-10:10,
samples=2000,
width=12cm,
xmin=0, xmax=1,
ymin=-10,ymax=10,
ytick={0},
yticklabels={0},
minor tick num=1,
%axis x line=center,
%axis y line=center,
xlabel=$\epsilon_r$,
use background,
]
% Función tangente
\addplot[blue] gnuplot[id=tangenvar,domain=0:1] {tan(11.45575*sqrt(2*x))};
% Función otra
\addplot[red] gnuplot[id=tangen,domain=0:1] {2*x*sqrt((1-x)/x)/(2*x-1)};
\legend{$\tg(l_r\sqrt{2\epsilon_r)}$,
  $\frac{2\epsilon_r}{2\epsilon_r-1}\sqrt{\frac{1-\epsilon_r}{\epsilon_r}}$}

% Asíntotas
\foreach \i in {0,0.5}
{\addplot[gray!50,dashed] coordinates{(\i,-10) (\i,10)};};%
\foreach \i in {pi/2,3*pi/2,5*pi/2,7*pi/2,9*pi/2}
{\addplot[gray!50,dashed] coordinates{(0.5*(\i/11.45575)^2,-10)(0.5*(\i/11.45575)^2,10)};};%

\end{axis}
\end{tikzpicture}
\caption{Los puntos de corte representan los valores propios $\epsilon_r$}
\label{fig:eigenvalues}
\end{figure}

\section{Funciones propias}
Las funciones de ondas (una por cada valor propio) en las distintas zonas se
calculó en ~\ref{wavefunctions}:
\[
\psi_r(x_r)=
\begin{cases}
A\,e^{\sqrt{2(1-\epsilon_r)}\,x_r}
& -\infty<x_r<0\\
A\,\left[\cos\left(\sqrt{2\epsilon_r}\,x_r\right)
  + \sqrt{\frac{1-\epsilon_r}{\epsilon}}\,
    \sen\left(\sqrt{2\epsilon_r}\,x_r\right)\right]
& 0<x_r<l_r\phantom{\rule[-10pt]{0em}{0em}}\\
A\,\left[\cos\left(l_r\sqrt{2\epsilon_r}\right)
  + \sqrt{\frac{1-\epsilon_r}{\epsilon}}\,
    \sen\left(l_r\sqrt{2\epsilon_r}\right)\right]
    e^{-\sqrt{2(1-\epsilon_r)}\,(x_r-l_r)}
& l_r<x_r<+\infty
\end{cases}
\]

donde $A$ se calcula:
\[
A = \frac{1}{\sqrt{I_1 + I_2 + I_3}}
\]

y las constantes $I_i$ ($i=1,2,3$):
\begin{align*}
&I_1 = \frac{1}{2\sqrt{2(1-\epsilon_r)}}\\
&I_2 
= 
\left\lbrace
l_r
+
\frac{1-2\epsilon_r}{4\epsilon_r\sqrt{2\epsilon_r}}\,
\left[
2l_r\sqrt{2\epsilon_r}
- \sen\left(2l_r\sqrt{2\epsilon_r}\right)
\right]
+
\frac{1 - \cos\left(2l_r\sqrt{2\epsilon_r}\right)}
{2\sqrt{2\epsilon_r}}\,
\sqrt{\frac{1-\epsilon_r}{\epsilon_r}}
\right\rbrace\\
&I_3 = \frac{1}{2\sqrt{2(1-\epsilon_r)}}
\end{align*}



\appendix
\chapter{First Appendix}

\end{document}

