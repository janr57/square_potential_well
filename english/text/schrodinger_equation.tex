\chapter{Schrödinger Equation}

What if we use dimensionless variables? Then we shall need a dimensionless
Scrödinger equation. Read on.

\section{Dimensionless variables}
We are going to use a dimensionless reduced position, $x_r$, defined as:
\begin{equation}
\label{reduced_x}
	x_r = \frac{x}{B}
\end{equation}
$B$ is a constant with length dimensions.

Likewise, we need a reduced energy, $\epsilon_r$:
\begin{equation}
\label{reduced_e}
	\epsilon_r = \frac{E}{C}
\end{equation}
where $C$ is another constant that has dimensions of energy.

Constants $A$ and $B$ must depend on values appearing in the Schrödinger
equation in one dimension, $x$:
\begin{equation}
\label{schro1}
-\frac{\hbar^2}{2m}\frac{d\psi(x)}{dx^2}+V(x)\psi(x) = E\psi(x)
\end{equation}

These values are the reduced Planck's constant, $\hbar$, the mass of the
particle, $m$, and the value of the potential surrounding the well, $V_0$.
Their respective dimensions are, mass, action and energy:
\begin{align*}
&[\hbar] = ML^2T^{-1}\\
&[m] = M\\
&[V_0] = ML^2T^{-2}
\end{align*}

Constant $B$ is a combination of these constants:
\[
[B] = \hbar^a m^b V_0^c =
(ML^2T^{-1})^a (M)^b (ML^2T^{-2})^c =
M^{a+b+c} L^{2a+2c} T^{-a-2c}
\]

The exponents $a$, $b$ and $c$ must be calculated knowing that $B$ has only
length dimensions ($M^0 L^1 T^0$).
Solving the corresponding system of equations:
\begin{align*}
&a + b + c = 0\\
&2a+2c = 1\\
&-a-2c = 0
\end{align*}

Solutions are $a=1$ and $b=c=-1/2$. Thus, $B$ is:
\begin{equation}
B = \hbar (mV_0)^{-1/2} = \frac{\hbar}{\sqrt{mV_0}}
\end{equation}

Constant $C$ is easier: $C=V_0$ as they both have energy dimensions.

So, we have:
\begin{equation}
\label{reduced_x2}
x_r = \frac{x}{B}=\frac{x}{\hbar/\sqrt{mV_0}} = \frac{\sqrt{mV_0}}{\hbar}\,x
\end{equation}

and
\begin{equation}
\label{reduced_e2}
\epsilon_r = \frac{E}{C}=\frac{E}{V_0}
\end{equation}

\subsection{Reduced wavefunction and Schrödinger equation}
Note that $|\psi(x)|^2\,dx$ is a probability, so it has no dimensions. Then,
the wavefunction, $\psi(x)$, must have $L^{-1/2}$ dimensions.

We can define a reduced wavefunction, $\psi_r$ as:
\begin{equation}
\label{reduced_psi}
\psi_r = \frac{\psi}{B^{-1/2}}
\end{equation}

Now we are going to get the dimensionless Schrödinger equation from the
normal one. Transforming \ref{schro1}:
\begin{equation}
\label{schro2}
\frac{\hbar^2}{2m}\frac{d^2\psi(x)}{dx^2} + (E-V(x))\psi(x) = 0
\end{equation}

Let's calculate $d\psi(x)/dx$ using \ref{reduced_psi}:
\[
\frac{d\psi(x)}{dx}
= \frac{d}{dx}\, B^{-1/2}\psi_r(x_r)
= B^{-1/2}\,\frac{d\psi_r(x_r)}{dx}
= B^{-1/2}\,\frac{d\psi_r(x_r)}{dx_r}\,\frac{dx_r}{dx}
\]

From \ref{reduced_x} we can calculate $dx_r/dx=B^{-1}$, so:
\[
\frac{d\psi(x)}{dx}
= B^{-1/2}\,\frac{d\psi_r(x_r)}{dx_r}\,B^{-1}
= B^{-3/2}\,\frac{d\psi_r(x_r)}{dx_r}
\]

Now, let's calculate $d^2\psi(x)/dx^2$:
\begin{align*}
\frac{d^2\psi(x)}{dx^2}
&= \frac{d}{dx}\left[\frac{d\psi_r(x_r)}{dx}\right]
= \frac{d}{dx}\left[B^{-3/2}\,\frac{d\psi_r(x_r)}{dx_r}\right]
= B^{-3/2}\,\frac{d}{dx}\,\left[\frac{d\psi_r(x_r)}{dx_r}\right]\\
&= B^{-3/2}\,\frac{d}{dx_r}\,\left[\frac{d\psi_r(x_r)}{dx_r}\right]
	\,\frac{dx_r}{dx}
= B^{-3/2}\,\frac{d^2\psi_r(x_r)}{dx_r^2} \,B^{-1}\\
&= B^{-5/2}\,\frac{d^2\psi_r(x_r)}{dx_r^2}
\end{align*}

\subsection{Zones I and III reduced Schrödinger equation}
In zones I and III, the potential energy is the non-zero constant: $V(x) = V_0$.

We substitute into the Scrödinger equation \ref{schro2}, using the reduced
energy, \ref{reduced_e2}, and the reduced wavefunction definitions,
\ref{reduced_psi}:
\[
B^{-5/2}\,\frac{d^2\psi_r}{dx_r^2}
	 + \frac{2m}{\hbar^2}(\epsilon_rV_0-V_0)\,B^{-1/2}\psi_r=0
\]

We multiply both terms by $B^{1/2}$:
\[
B^{-5/2}B^{1/2}\,\frac{d^2\psi_r}{dx_r^2}
	 - \frac{2mV_0}{\hbar^2}(1-\epsilon_r)\,B^{-1/2}\,B^{1/2}\,\psi_r=0
\]

\[
B^{-2}\,\frac{d^2\psi_r}{dx_r^2}
	 + \frac{2mV_0}{\hbar^2}(\epsilon_r-1)\,\psi_r=0
\]

From \ref{reduced_x2}, we know that $B^{-2}=mV_0/\hbar^2$:
\[
\frac{mV_0}{\hbar^2}\,\frac{d^2\psi_r}{dx_r^2}
	 - \frac{2mV_0}{\hbar^2}(1-\epsilon_r)\,\psi_r=0
\]

We get the zones I and III reduced Schrödinger equation:
\begin{equation}
\frac{d^2\psi_r}{dx_r^2} - 2\,(1-\epsilon_r)\,\psi_r=0
\end{equation}

\subsection{Zone II reduced Schrödinger equation}
In zone II, the potential energy is zero: $V(x) = 0$.
The Schrodinger equation is:
\[
\label{schro2}
\frac{\hbar^2}{2m}\frac{d^2\psi(x)}{dx^2} + E\,\psi(x) = 0
\]

As we did in the previous subsection, we get the reduced Schrödinger equation:
\begin{equation}
\frac{d^2\psi_r}{dx_r^2} + 2\epsilon_r\,\psi_r=0
\end{equation}


\section{Bound states}
Recall that bonded states are those in which a particle is inside the
potential well, so it has less total energy than the potential energy in its
surroundings, so it wouldn't escape classically.

\subsection{Initial solving}
\subsubsection{Zone I}
Domain: $x_r < 0$.\\
Reduced Schrödinger equation: $\psi''_{rI} - 2(1-\epsilon_r)\psi_{rI} = 0$.\\
Characteristic equation using $\psi_{rI}=e^{sx_r}$:
\[
s^2-2(1-\epsilon_r)=0
\]
\[
s^2=2(1-\epsilon_r)
\]
\[
s = \pm\sqrt{2(1-\epsilon_r)}
\]
So:
\[
\psi_{rI}(x_r)
 = C_1\,e^{-\sqrt{2(1-\epsilon_r)}\,x_r} + C_2\,e^{\sqrt{2(1-\epsilon_r)}\,x_r}
\]
Taking into account the following boundary condition:
\[
\lim_{\scriptsize x_r\to\, -\infty} \psi_{rI}(x_r) = 0
\]

we deduce that $C_1$ must be zero. The zone I wavefunction is:
\[
\psi_{rI}(x_r) = C_2\,e^{\sqrt{2(1-\epsilon_r)}\,x_r}
\]

\subsubsection{Zone II}
Domain: $0 < x_r < l_r$.\\
Reduced Schrödinger equation: $\psi''_{rII} - 2\epsilon_r\psi_{rII}= 0$.\\
Characteristic equation using $\psi_{rII}=e^{sx_r}$:
\[
s^2+2\epsilon_r=0
\]
\[
s^2=-2\epsilon_r
\]
\[
s = \pm i\sqrt{2\epsilon_r}
\]
So:
\begin{align*}
\psi_{rII}(x_r)
&=C_3\,e^{-i\,\sqrt{2\epsilon_r}\,x_r} + C_4\,e^{+i\,\sqrt{2\epsilon_r}\,x_r}\\
&=C_3\,\cos\left(\sqrt{2\epsilon_r}\,x_r\right)
 - i C_3\,\sin\left(\sqrt{2\epsilon_r}\,x_r\right)
 + C_4\,\cos\left(\sqrt{2\epsilon_r}\right)\,x_r 
 + i C_4\,\sin\left(\sqrt{2\epsilon_r}\,x_r\right)\\
&=(C_3+C_4)\,\cos\left(\sqrt{2\epsilon_r}\,x_r\right)
 + i (C_4-C_3)\,\sin\left(\sqrt{2\epsilon_r}\,x_r\right)
\end{align*}

we deduce that $C_6$ must be zero. The zone I wavefunction is:
\[
\psi_{rII}(x_r)
= A\,\cos\left(\sqrt{2\epsilon_r}\,x_r\right)
 + B\,\sin\left(\sqrt{2\epsilon_r}\,x_r\right)
\]

\subsubsection{Zone III}
Domain: $x_r > l_r$.\\
Reduced Schrödinger equation: $\psi''_{rIII} - 2(1-\epsilon_r)\psi_{rIII}= 0$.\\
Characteristic equation using $\psi_{rIII}=e^{sx_r}$:
\[
s^2-2(1-\epsilon_r)=0
\]
\[
s^2=2(1-\epsilon_r)
\]
\[
s = \pm\sqrt{2(1-\epsilon_r)}
\]
So:
\[
\psi_{rIII}(x_r)
 = C_5\,e^{-\sqrt{2(1-\epsilon_r)}\,x_r} + C_6\,e^{\sqrt{2(1-\epsilon_r)}\,x_r}
\]
Taking into account the following boundary condition:
\[
\lim_{\scriptsize x_r\to\, +\infty} \psi_{rIII}(x_r) = 0
\]

we deduce that $C_6$ must be zero. The zone I wavefunction is:
\[
\psi_{rIII}(x_r) = C_5\,e^{\sqrt{2(1-\epsilon_r)}\,x_r}
\]

\subsection{Other boundary conditions}

Wavefunctions I and II must tend to the same value at $x_r=0$:
\[
\lim_{\scriptsize x_r\to\,0} \psi_{rI}(x_r)
=
\lim_{\scriptsize x_r\to\,0} \psi_{rII}(x_r)
\]
So:
\[
C_2 = A
\]

The derivatives of wavefunctions I and II must tend to the same value at
 $x_r=0$:
\[
\lim_{\scriptsize x_r\to\,0} \psi'_{rI}(x_r)
=
\lim_{\scriptsize x_r\to\,0} \psi'_{rII}(x_r)
\]
Let's calculate the derivatives of the wavefunctions:
\begin{align*}
\psi'_{rI} &= \frac{d\psi_{rI}(x_r)}{dx_r}
  = C_2\,\sqrt{2(1-\epsilon_r)}\,e^{\sqrt{2(1-\epsilon_r)}\,x_r}
  = A\,\sqrt{2(1-\epsilon_r)}\,e^{\sqrt{2(1-\epsilon_r)}\,x_r}\\
\psi'_{rII} &= \frac{d\psi_{rII}(x_r)}{dx_r}
  = -A\,\sqrt{2\epsilon_r}\,\sin\left(\sqrt{2\epsilon_r}\,x_r\right)
  + B\,\sqrt{2\epsilon_r}\,\cos\left(\sqrt{2\epsilon_r}\,x_r\right)
\end{align*}

Then:
\[
A\,\sqrt{2(1-\epsilon_r)} =  B\,\sqrt{2\epsilon_r}
\]
\[
B = \sqrt{\frac{1-\epsilon_r}{\epsilon_r}}\,A
\]

Wavefunctions II and III must tend to the same value at $x_r=l_r$:
\[
\lim_{\scriptsize x_r\to\,l_r} \psi_{rII}(x_r)
=
\lim_{\scriptsize x_r\to\,l_r} \psi_{rIII}(x_r)
\]
\[
A\,\cos\left(l_r\sqrt{2\epsilon_r}\right)
 + A\,\sqrt{\frac{1-\epsilon_r}{\epsilon_r}}
   \,\sin\left(l_r\,\sqrt{2\epsilon_r}\right)
=
C_5\,\,e^{-l_r\sqrt{2(1-\epsilon_r}}
\]

Solving for $C_5$:
\begin{equation}
\label{c5_1}
C_5 = A\,e^{l_r\sqrt{2(1-\epsilon_r)}}
  \left\lbrace\cos\left(l_r\sqrt{2\epsilon_r}\right) 
  + \sqrt{\frac{1-\epsilon_r}{\epsilon_r}}\,
         \sin\left(l_r\sqrt{2\epsilon_r}\right)\right\rbrace
\end{equation}

\subsection{Eigenvalue equation}
The derivatives of the zone II and zone III wavefunctions must have the same
limit at $l_r$:
\[
\lim_{\scriptsize x_r\to\,l_r} \psi'_{rII}(x_r)
=
\lim_{\scriptsize x_r\to\,l_r} \psi'_{rIII}(x_r)
\]

The derivative of $\psi_{rIII}$ is:
\[
\psi'_{rIII}(x_r) = \frac{d\psi_{rIII}(x_r)}{dx_r}
  = -C_5\,\sqrt{2(1-\epsilon_r)}\,e^{-\sqrt{2(1-\epsilon_r)}\,x_r}
\]

Then, we get:
\[
-A\,\sqrt{2\epsilon_r}\,\sin\left(l_r\sqrt{2\epsilon_r}\right)
 + A\,\sqrt{\frac{1-\epsilon_r}{\epsilon_r}}\,\sqrt{2\epsilon_r}
   \,\cos\left(l_r\,\sqrt{2\epsilon_r}\right)
=
-C_5\,\sqrt{2(1-\epsilon_r)}\,e^{-l_r\sqrt{2(1-\epsilon_r)}}
\]

Solving for $C_5$:
\begin{equation}
\label{c5_2}
C_5 = A\,e^{l_r\sqrt{2(1-\epsilon_r)}}\,\sqrt{\frac{\epsilon_r}{1-\epsilon_r}}
  \left\lbrace\sin\left(l_r\sqrt{2\epsilon_r}\right)\,
  - \sqrt{\frac{1-\epsilon_r}{\epsilon_r}}\,
       \cos\left(l_r\sqrt{2\epsilon_r}\right)\right\rbrace
\end{equation}

Expressions \ref{c5_1} and \ref{c5_2} are equal. Simplifying:
\[
\cos\left(l_r\sqrt{2\epsilon_r}\right) 
  + \sqrt{\frac{1-\epsilon_r}{\epsilon_r}}\,
         \sin\left(l_r\sqrt{2\epsilon_r}\right)
=
\sqrt{\frac{\epsilon_r}{1-\epsilon_r}}
  \left\lbrace\sin\left(l_r\sqrt{2\epsilon_r}\right)\,
  - \sqrt{\frac{1-\epsilon_r}{\epsilon_r}}\,
       \cos\left(l_r\sqrt{2\epsilon_r}\right)\right\rbrace
\]

Moving left the square root which is next to the equal sign:
\[
\sqrt{\frac{1-\epsilon_r}{\epsilon_r}}
\cos\left(l_r\sqrt{2\epsilon_r}\right)\,
  + \frac{1-\epsilon_r}{\epsilon_r}\,
         \sin\left(l_r\sqrt{2\epsilon_r}\right)
=
  \sin\left(l_r\sqrt{2\epsilon_r}\right)\,
  - \sqrt{\frac{1-\epsilon_r}{\epsilon_r}}\,
       \cos\left(l_r\sqrt{2\epsilon_r}\right)
\]

Moving cosines to the left and sines to the right:
\[
2\sqrt{\frac{1-\epsilon_r}{\epsilon_r}}
\cos\left(l_r\sqrt{2\epsilon_r}\right) 
=
  \sin\left(l_r\sqrt{2\epsilon_r}\right)\,
  - \frac{1-\epsilon_r}{\epsilon_r}\,
       \sin\left(l_r\sqrt{2\epsilon_r}\right)
\]

\[
2\sqrt{\frac{1-\epsilon_r}{\epsilon_r}}
\cos\left(l_r\sqrt{2\epsilon_r}\right) 
=
  \left[1-\frac{1-\epsilon_r}{\epsilon_r}\right]
  \sin\left(l_r\sqrt{2\epsilon_r}\right)
\]

\[
2\sqrt{\frac{1-\epsilon_r}{\epsilon_r}}
\cos\left(l_r\sqrt{2\epsilon_r}\right) 
=
  \frac{2\epsilon_r-1}{\epsilon_r}
  \sin\left(l_r\sqrt{2\epsilon_r}\right)
\]

The eigenvalue equation is:
\begin{equation}
\label{eigenvalue_eq}
\tan\left(l_r\sqrt{2\epsilon_r}\right)
=
\frac{2\epsilon_r}{2\epsilon_r-1}
\sqrt{\frac{1-\epsilon_r}{\epsilon_r}}
\end{equation}
This equation cannot be solved exactly, but we could use numerical methods.
See the numerical solution of this equation later.

\subsection{Eigenfunctions}
Summing up, taking into account the possible eigenvalues $\epsilon_r$, the
eigenfunction at every zone is:
\begin{equation}
\psi_r(x_r)=
\begin{cases}
A\,e^{\sqrt{2(1-\epsilon_r)}\,x_r}
& -\infty<x_r<0\\
A\,\left[\cos\left(\sqrt{2\epsilon_r}\,x_r\right)
  + \sqrt{\frac{1-\epsilon_r}{\epsilon}}\,
    \sin\left(\sqrt{2\epsilon_r}\,x_r\right)\right]
& 0<x_r<l_r\phantom{\rule[-9pt]{0em}{0em}}\\
A\,\left[\cos\left(l_r\sqrt{2\epsilon_r}\right)
  + \sqrt{\frac{1-\epsilon_r}{\epsilon}}\,
    \sin\left(l_r\sqrt{2\epsilon_r}\right)\right]
    e^{-\sqrt{2(1-\epsilon_r)}\,(x_r-l_r)}
& l_r<x_r<+\infty
\end{cases}
\end{equation}

\subsection{Eigenfunctions squares}
We'll calculate the squares of the eigenfunctions to normalize them:

\subsubsection{Zone I}
\[
\left|\psi_{rI}(x_r)\right|^2
= A^2\left(e^{\sqrt{2(1-\epsilon_r)}\,x_r}\right)^2
= A^2\,e^{2\sqrt{2(1-\epsilon_r)}\,x_r}
\]

\subsubsection{Zone II}
\begin{align*}
\left|\psi_{rII}(x_r)\right|^2
&= A^2\,\left\lbrace\cos\left(\sqrt{2\epsilon_r}\,x_r\right)
  + \sqrt{\frac{1-\epsilon_r}{\epsilon_r}}\,
    \sin\left(\sqrt{2\epsilon_r}\,x_r\right)
\right\rbrace^2\\
&= A^2\,\left\lbrace
 \cos^2\left(\sqrt{2\epsilon_r}\,x_r\right)\,
 + \frac{1-\epsilon_r}{\epsilon_r}\,\sin^2\left(\sqrt{2\epsilon_r}\,x_r\right)
 + 2\sqrt{\frac{1-\epsilon_r}{\epsilon_r}}\,
 \sin\left(\sqrt{2\epsilon_r}\,x_r\right)\,
 \cos\left(\sqrt{2\epsilon_r}\,x_r\right)
\right\rbrace\\
&= A^2\,\left\lbrace
 1 - \sin^2\left(\sqrt{2\epsilon_r}\,x_r\right)\,
 + \frac{1-\epsilon_r}{\epsilon_r}\,\sin^2\left(\sqrt{2\epsilon_r}\,x_r\right)
 + \sqrt{\frac{1-\epsilon_r}{\epsilon_r}}\,
 \sin\left(2\sqrt{2\epsilon_r}\,x_r\right)
\right\rbrace\\
&= A^2\,\left\lbrace
 1  
 + \left(-1+\frac{1-\epsilon_r}{\epsilon_r}\right)\,\sin^2\left(\sqrt{2\epsilon_r}\,x_r\right)
 + \sqrt{\frac{1-\epsilon_r}{\epsilon_r}}\,
 \sin\left(2\sqrt{2\epsilon_r}\,x_r\right)
\right\rbrace\\
&= A^2\,\left\lbrace
 1  
 + \frac{1-2\epsilon_r}{\epsilon_r}\,\sin^2\left(\sqrt{2\epsilon_r}\,x_r\right)
 + \sqrt{\frac{1-\epsilon_r}{\epsilon_r}}\,
 \sin\left(2\sqrt{2\epsilon_r}\,x_r\right)
\right\rbrace
\end{align*}

\subsubsection{Zone III}

\begin{align*}
\left|\psi_{rIII}(x_r)\right|^2
&= A^2\,\left\lbrace\cos\left(l_r\sqrt{2\epsilon_r}\right)
  + \sqrt{\frac{1-\epsilon_r}{\epsilon_r}}\,
    \sin\left(l_r\sqrt{2\epsilon_r}\right)
\right\rbrace^2
\left\lbrace
\,e^{-\sqrt{2(1-\epsilon_r)}\,(x_r-l_r)}
\right\rbrace^2\\
&= A^2\,\left\lbrace
 \cos^2\left(l_r\sqrt{2\epsilon_r}\right)\,
 + \frac{1-\epsilon_r}{\epsilon_r}\,\sin^2\left(l_r\sqrt{2\epsilon_r}\right)
 + \sqrt{\frac{1-\epsilon_r}{\epsilon_r}}\,
 \sin\left(2l_r\sqrt{2\epsilon_r}\right)
\right\rbrace
\,e^{-2\sqrt{2(1-\epsilon_r)}\,(x_r-l_r)}\\
&= A^2\,\left\lbrace
 1 - \sin^2\left(l_r\sqrt{2\epsilon_r}\right)\,
 + \frac{1-\epsilon_r}{\epsilon_r}\,\sin^2\left(l_r\sqrt{2\epsilon_r}\right)
 + \sqrt{\frac{1-\epsilon_r}{\epsilon_r}}\,
 \sin\left(2l_r\sqrt{2\epsilon_r}\right)
\right\rbrace
\,e^{-2\sqrt{2(1-\epsilon_r)}\,(x_r-l_r)}\\
&= A^2\,\left\lbrace
 1 
 + \left(-1+\frac{1-\epsilon_r}{\epsilon_r}\right)\,\sin^2\left(l_r\sqrt{2\epsilon_r}\right)
 + \sqrt{\frac{1-\epsilon_r}{\epsilon_r}}\,
 \sin\left(2l_r\sqrt{2\epsilon_r}\right)
\right\rbrace
\,e^{-2\sqrt{2(1-\epsilon_r)}\,(x_r-l_r)}\\
&= A^2\,\left\lbrace
 1 
 + \frac{1-2\epsilon_r}{\epsilon_r}\,\sin^2\left(l_r\sqrt{2\epsilon_r}\right)
 + \sqrt{\frac{1-\epsilon_r}{\epsilon_r}}\,
 \sin\left(2l_r\sqrt{2\epsilon_r}\right)
\right\rbrace
\,e^{-2\sqrt{2(1-\epsilon_r)}\,(x_r-l_r)}\\
&= A^2\,e^{-2\sqrt{2(1-\epsilon_r)}\,(x_r-l_r)}
\end{align*}

Regarding the last step, note that:
\[
 \frac{1-2\epsilon_r}{\epsilon_r}\,\sin^2\left(l_r\sqrt{2\epsilon_r}\right)
 + \sqrt{\frac{1-\epsilon_r}{\epsilon_r}}\,
 \sin\left(2l_r\sqrt{2\epsilon_r}\right) = 0
\]

only if $\epsilon_r$ is an eigenvalue.

Let's demonstrate this. Supposing that $\epsilon_r$ is an eigenvalue, then
it must comply the eigenvalue equation, \ref{eigenvalue_eq}:
\[
\tan\left(l_r\sqrt{2\epsilon_r}\right)
=
\frac{2\epsilon_r}{2\epsilon_r-1}
\sqrt{\frac{1-\epsilon_r}{\epsilon_r}}
\]

\[
\frac{\sin\left(l_r\sqrt{2\epsilon_r}\right)}
{\cos\left(l_r\sqrt{2\epsilon_r}\right)}
=
\frac{2\epsilon_r}{2\epsilon_r-1}
\sqrt{\frac{1-\epsilon_r}{\epsilon_r}}
\]

Rearranging terms:
\[
\frac{2\epsilon_r-1}{\epsilon_r}\,
\sin\left(l_r\sqrt{2\epsilon_r}\right)
=
2\sqrt{\frac{1-\epsilon_r}{\epsilon_r}}
\cos\left(l_r\sqrt{2\epsilon_r}\right)
\]

\[
\frac{1-2\epsilon_r}{\epsilon_r}\,
\sin\left(l_r\sqrt{2\epsilon_r}\right)
+
2\sqrt{\frac{1-\epsilon_r}{\epsilon_r}}
\cos\left(l_r\sqrt{2\epsilon_r}\right)
=0
\]

Multiplying both terms by the sine of the angle:
\[
\frac{1-2\epsilon_r}{\epsilon_r}\,
\sin^2\left(l_r\sqrt{2\epsilon_r}\right)
+
2\sqrt{\frac{1-\epsilon_r}{\epsilon_r}}
\sin\left(l_r\sqrt{2\epsilon_r}\right)\,
\cos\left(l_r\sqrt{2\epsilon_r}\right)
=0
\]

Then we demonstrate that:
\[
\frac{1-2\epsilon_r}{\epsilon_r}\,
\sin^2\left(l_r\sqrt{2\epsilon_r}\right)
+
\sqrt{\frac{1-\epsilon_r}{\epsilon_r}}
\sin\left(2l_r\sqrt{2\epsilon_r}\right)
=0
\]

\subsection{Normalization}
The wavefunction must be normalized. That is:
\[
\int_{-\infty}^{+\infty} \left|\psi_r(x_r)\right|^2 dx_r = 1
\]

\[
\int_{-\infty}^{0}\left|\psi_{rI}(x_r)\right|^2 dx_r 
 + \int_{0}^{l_r}\left|\psi_{rII}(x_r)\right|^2 dx_r 
 + \int_{l_r}^{+\infty}\left|\psi_{rIII}(x_r)\right|^2 dx_r 
 = 1
\]

We must calculate three different integrals:
\begin{equation}
\label{normalization_eq}
I_1 + I_2 + I_3 = 1
\end{equation}

\subsubsection{First integral: $I_1$}

\[
I_1 = \int_{-\infty}^{0}\left|\psi_{rI}(x_r)\right|^2 dx_r 
= 
A^2\,\int_{-\infty}^0 e^{2\sqrt{2(1-\epsilon_r)}\,x_r} dx_r
\]

Change of variable:
\[
2\sqrt{2(1-\epsilon_r)}\,x_r = t_r
\hspace{2em}
dx_r = \frac{1}{2\sqrt{2(1-\epsilon_r)}}\,dt_r
\]

\[
I_1
= A^2\,\frac{1}{2\sqrt{2(1-\epsilon_r)}}\int_{-\infty}^0 e^{t_r} dt_r
= A^2\,\frac{1}{2\sqrt{2(1-\epsilon_r)}}(e^0-e^{-\infty})
\]

Then:
\begin{equation}
\label{I1}
I_1 = \frac{A^2}{2\sqrt{2(1-\epsilon_r)}}
\end{equation}

\subsubsection{Third integral: $I_3$}

We skip the second one for the moment:
\[
I_3 = \int_{l_r}^{+\infty}\left|\psi_{rIII}(x_r)\right|^2 dx_r 
= 
A^2\,\int_{l_r}^{+\infty} e^{-2\sqrt{2(1-\epsilon_r)}\,(x_r-l_r)} dx_r
\]

Change of variable:
\[
-2\sqrt{2(1-\epsilon_r)}\,(x_r-l_r) = t_r
\hspace{2em}
dx_r = \frac{-1}{2\sqrt{2(1-\epsilon_r)}}\,dt_r
\]

\[
I_3
= \frac{-A^2}{2\sqrt{2(1-\epsilon_r)}}\int_{0}^{-\infty} e^{t_r} dt_r
= \frac{-A^2}{2\sqrt{2(1-\epsilon_r)}}(e^{-\infty}-e^{0})
\]

Then:
\begin{equation}
\label{I3}
I_3 = \frac{A^2}{2\sqrt{2(1-\epsilon_r)}}
\end{equation}

\subsubsection{Second integral: $I_2$}

As we will see, this integral will be solved calculating three integrals,
$I_21$, $I_22$, $I_23$:

\[
I_2 = \int_{0}^{l_r}\left|\psi_{rII}(x_r)\right|^2 dx_r 
= 
A^2\,\int_{0}^{l_r}
\left\lbrace
1
 + \frac{1-2\epsilon_r}{\epsilon_r}\,\sin^2\left(\sqrt{2\epsilon_r}\,x_r\right)
 + \sqrt{\frac{1-\epsilon_r}{\epsilon_r}}\,\sin\left(2\sqrt{2\epsilon_r}\,x_r\right)
\right\rbrace
dx_r
\]

\[
I_2 = A^2\,\left\lbrace I_{21} + I_{22} + I_{23}\right\rbrace
\]

\subsubsection{Second integral, first part: $I_{21}$}

\[
I_{21}
= 
\int_{0}^{l_r} dx_r
\]

\begin{equation}
\label{I21}
I_{21} = l_r
\end{equation}

\subsubsection{Second integral, second part: $I_{22}$}

\[
I_{22}
= 
\frac{1-2\epsilon_r}{\epsilon_r}\,
\int_{0}^{l_r}\sin^2\left(\sqrt{2\epsilon_r}\,x_r\right) dx_r
\]

Change of variable:
\[
\sqrt{2\epsilon_r}\,x_r = t_r
\hspace{2em}
dx_r = \frac{1}{\sqrt{2\epsilon_r}}\,dt_r
\]

\[
I_{22}
= 
\frac{1-2\epsilon_r}{\epsilon_r}\,
\frac{1}{\sqrt{2\epsilon_r}}\,
\int_{0}^{l_r\sqrt{2\epsilon_r}}\hspace*{-2em}\sin^2 t_r\, dt_r
\]

Let's write $\sin^2 t_r$ in a different way:
\[
\cos 2t_r = \cos^2 t_r - \sin^2 t_r = 1-\sin^2 t_r - \sin^2 t_r
= 1-2\sin^2 t_r
\]

\[
\sin^2 t_r = \frac{1}{2} - \frac{1}{2}\,\cos (2t_r)
\]

\[
I_{22}
= 
\frac{1-2\epsilon_r}{\epsilon_r\sqrt{2\epsilon_r}}\,
\left\lbrace
\frac{1}{2}\int_{0}^{l_r\sqrt{2\epsilon_r}}\hspace*{-2em}\,dt_r
- \frac{1}{2}\int_{0}^{l_r\sqrt{2\epsilon_r}}\hspace*{-2em}\cos(2t_r)\, dt_r
\right\rbrace
\]

Change of variable:
\[
2t_r = z_r
\hspace{2em}
dt_r = \frac{1}{2}\,dz_r
\]

\begin{align*}
I_{22}
&= 
\frac{1-2\epsilon_r}{\epsilon_r\sqrt{2\epsilon_r}}\,
\left\lbrace
\frac{l_r\sqrt{2\epsilon_r}}{2}
- \frac{1}{4}\int_{0}^{2l_r\sqrt{2\epsilon_r}}\hspace*{-2em}\cos(2t_r)\, dt_r
\right\rbrace\\
&= 
\frac{1-2\epsilon_r}{\epsilon_r\sqrt{2\epsilon_r}}\,
\left\lbrace
\frac{2l_r\sqrt{2\epsilon_r}}{4}
- \frac{\sin\left(2l_r\sqrt{2\epsilon_r}\right)}{4}
\right\rbrace
\end{align*}

\begin{equation}
\label{I22}
I_{22} =
\frac{1-2\epsilon_r}{4\epsilon_r\sqrt{2\epsilon_r}}\,
\left\lbrace
2l_r\sqrt{2\epsilon_r}
- \sin\left(2l_r\sqrt{2\epsilon_r}\right)
\right\rbrace
\end{equation}

\subsubsection{Second integral, third part: $I_{23}$}

\[
I_{23}
= 
\sqrt{\frac{1-\epsilon_r}{\epsilon_r}}\,
\int_{0}^{l_r}\sin\left(2\sqrt{2\epsilon_r}\,x_r\right)\, dx_r
\]

Change of variable:
\[
2\sqrt{2\epsilon_r}\,x_r = t_r
\hspace{2em}
dx_r = \frac{1}{2\sqrt{2\epsilon_r}}\,dt_r
\]

\begin{align*}
I_{23}
&= 
\sqrt{\frac{1-\epsilon_r}{\epsilon_r}}\,
\frac{1}{2\sqrt{2\epsilon_r}}\,
\int_{0}^{2l_r\sqrt{2\epsilon_r}}\hspace*{-2em}\sin t_r\, dt_r\\
&= 
-\sqrt{\frac{1-\epsilon_r}{\epsilon_r}}\,
\frac{1}{2\sqrt{2\epsilon_r}}\,
\left[\cos\left(2l_r\sqrt{2\epsilon_r}\right) - cos 0\right]
\end{align*}

\begin{equation}
\label{I23}
I_{23} =
\sqrt{\frac{1-\epsilon_r}{\epsilon_r}}\,
\frac{1}{2\sqrt{2\epsilon_r}}\,
\left[1 - \cos\left(2l_r\sqrt{2\epsilon_r}\right)\right]
\end{equation}

So $I_2$ is:
\begin{equation}
\label{I2}
I_2 
= A^2\,
\left\lbrace
l_r
+
\frac{1-2\epsilon_r}{4\epsilon_r\sqrt{2\epsilon_r}}\,
\left[
2l_r\sqrt{2\epsilon_r}
- \sin\left(2l_r\sqrt{2\epsilon_r}\right)
\right]
+
\sqrt{\frac{1-\epsilon_r}{\epsilon_r}}\,
\frac{1}{2\sqrt{2\epsilon_r}}\,
\left[1 - \cos\left(2l_r\sqrt{2\epsilon_r}\right)\right]
\right\rbrace
\end{equation}

From the normalization equation, \ref{normalization_eq}, we get the
constant $A$:
\[
A = \frac{1}{\sqrt{I_1 + I_2 + I_3}}
\]
